\chapter{Topology}

\section{Topological spaces}
\label{sec:topological_spaces}

\begin{dfn}
  Let $X$ be a set. A \emph{topology on $X$} is a subset of the
  power set of $X$, whose elements are called \emph{open subsets of the topology}
  and which fulfills the following axioms:
  
  \paragraph{Union}
  Let $(U_i)_{i \in I}$ be a family of open subsets of the topology. Then the
  union $\bigcup_{i \in I} U_i$ is again an open subset of the topology.
  
  \paragraph{Intersection}
  Let $U_1$, \dots, $U_n$ be finitely many open subsets of the topology. Then
  their intersection $U_1 \cap \dotsb \cap U_n$
  is again an open subset of the topology.
  
  A \emph{topological space $X$} is a set $X$ together with a topology $\mathfrak U^0(X)$
  on $X$. An \emph{open set of a topological space $X$} is an open set of the topology
  of $X$.
  A \emph{point $p \in X$ of a topological space $X$} is an element of the
  underlying set of $X$.
\end{dfn}

As the union of an empty family of subsets of a set $X$ is the empty subset,
the empty subset is open with respect to any topology on $X$.
As the intersection of an empty family of subsets of $X$ is the whole set $X$,
the whole set $X$ is open with respect to any topology on $X$.

Let $X$ be a topological space.
A \emph{neighborhood of a point $p$ in $X$} is a subset $G$
of the underlying set of $X$ such that there exists an open subset $U$ of $X$ with
$p \in U \subseteq G$. The set of neighborhoods of $p$ is denoted by
$\mathfrak U(p, X)$. A neighborhood that is at the same time an open subset
is an \emph{open neighborhood}. The set of open neighborhoods of $p$ is
denoted by $\mathfrak U^0(p, X)$. A subset $U$ of the underlying set of the
topological space $X$ is open if and only if it is a neighborhood of every point
$p \in U$. A \emph{neighborhood of a subset $A$ of $X$} is a subset $G$ that is
a neigborhood of every point $p \in A$. The set of neighborhoods of $A$ is denoted
by $\mathfrak U(A, X)$.

An \emph{closed subset $Z \subseteq X$ of $X$} is a subset of the underlying
set of $X$ such that its complement $X \setminus Z$ is open. Let $Y$ be an
arbitrary subset of $X$. The \emph{closure $\overline Y$ of $Y$ in $X$} is the
smallest closed subset of $X$ containing $Y$, that is
\begin{equation}
  \overline Y = \bigcap_{\text{$Y \subseteq Z \subseteq X$ is a closed subset}} Z
\end{equation}

Let $Y$ be any subset of the underlying set of $X$. Set
\[
  \mathfrak U^0(Y) \coloneqq \Set{Y \cap U: U \in \mathfrak U^0(X)}
\]
This is a topology, the \emph{subspace topology}. Endowed with this topology,
$Y$ canonically becomes a topological space itself. Topological spaces of this form
are called \emph{subspaces of $X$}. If $Y$ is a closed subset of $X$, a subset of
$Y$ is closed in $Y$ if and only if it is closed in $X$. In particular, the closure
of a subset of $Y$ in $Y$ is the same as its closure in $X$.

The cartesian space $\set R^n$ is canonically a topological space with the
topology defined by
\begin{equation}
  \label{eq:cartesian_topology}
  \mathfrak U^0(\set R^n) = \Set{U \subseteq \set R^n : \forall p \in U \,
  \exists \epsilon \in \set R^+: U_\epsilon(p) \subseteq U},
\end{equation}
where
\[
  U_\epsilon(p) \coloneqq \Set{q \in \set R^n : \norm{q - p} < \epsilon}
\]
for an $\epsilon > 0$ is the standard \emph{$\epsilon$-neighborhood}. Its 
closure under the so-defined topology is given by
\[
  \overline U_{\epsilon(p)} = B_\epsilon(p) \coloneqq
  \Set{q \in \set R^n : \norm {q - p} \leq \epsilon}.
\]

Given a pair of topological space $X$ and $Y$, the product
$X \times Y = \Set{(x, y) : x \in X, y \in Y}$ carries a canonical topology
defined by
\begin{equation}
  \label{eq:product_topology}
  \mathfrak U^0(X \times Y) = \Set{W \subseteq X \times Y \forall (x, y) \in W \,
  \exists U \in \mathfrak U^0(x, X), V \in \mathfrak U^0(y, Y):
  U \times V \subseteq W}.
\end{equation}
Endowed with this topology, $X \times Y$ is called the \emph{product space} or
\emph{topological product of $X$ and $Y$}. If $A$ is a subspace of $X$ and
$B$ is a subspace of $Y$, the subspace $A \times B$ of $X \times Y$ carries the
product topology of the topological product between $A$ and $B$.

\section{Continuous maps}
\label{sec:continuity}

\begin{dfn}
  A \emph{continuous map $f\colon X \to Y$ from an topological space $X$ to
  a topological space $Y$} is a mapping $f\colon X \to Y$ such that
  \[
    \forall V \in \mathfrak U^0(Y) : f^{-1}(V) \in \mathfrak U^0(X).
  \]
\end{dfn}

Let $X$ be a topological space and let $\phi\colon X \to \set R$ be a continuous map.
The \emph{support $\supp \phi$ of $\phi$} is the closed subset
\[
  \Set{p \in X : \forall G \in \mathfrak U(p, X): \phi|G \not\equiv 0}.
\]
It is given by
\begin{equation}
  \label{eq:support}
  \supp \phi = \overline{\Set{p \in X : \phi(p) \neq 0}}.
\end{equation}

If $X$ and $Y$ are two topological spaces, the two projection maps
$\pr_1\colon X \times Y \to X, (x, y) \mapsto x$ and
$\pr_2\colon X \times Y \to Y, (x, y) \mapsto y$ are continuous.

\section{Hausdorff spaces}
\label{sec:hausdorff_spaces}

\begin{dfn}
  A \emph{Hausdorff space $X$} is a topological space $X$ such that for every
  pair $p$ and $q$ of points of $X$ with $p \neq q$, there are neighborhoods $G$
  of $p$ and $H$ of $q$, respectively, such that $G \cap H = \emptyset$.
\end{dfn}
A subspace of a Hausdorff space is again a Hausdorff space. Any finite subset of
a Hausdorff space is a closed subset. The topological product of two Hausdorff
spaces is again a Hausdorff space.

Cartesian space $\set R^n$ is an example for a Hausdorff space.

\section{Normal spaces}

\begin{dfn}
  A Hausdorff space $X$ is \emph{normal} such that for any disjoint closed
  subsets $A$ and $B$ of $X$ there exists neighborhoods $U \in \mathfrak U(A, X)$
  and $V \in \mathfrak U(B, X)$ with $U \cap V = \emptyset$.
\end{dfn}

An open cover $(U_i)_{i \in I}$ of a topological space $X$ is \emph{locally finite}
if every point of $X$ possesses a neighborhood $G$ such that there only finitely
many $i \in I$ with $G \cap U_i \neq \emptyset$.
\begin{thm}[Shrinking lemma]
  \label{prop:shrinking_lemma}
  For any locally finite open cover $(U_i)_{i \in I}$ of a normal Hausdorff space $X$, there
  exists a locally finite open cover $(V_i)_{i \in I}$ with
  \[
    \forall i \in I : V_i \subseteq \overline{V_i} \subseteq U_i.
  \]
\end{thm}

\section{Compact spaces}
\label{sec:compact}

A family $(U_i)_{i \in I}$ of open subsets of a topological space $X$ is an
\emph{open cover of $X$} if
\[
  X = \bigcup_{i \in I} U_i.
\]
A \emph{subcover} of an open cover $(U_i)_{i \in I}$ of $X$
is an open cover of the form $(U_i)_{i \in I'}$ for $I' \subseteq I$. An open
cover $(U_i)_{í \in I}$ of $X$ is \emph{finite} if the index set $I$ is finite.

If $Y$ is a subspace of a topological space $X$ and $(U_i)_{i \in I}$ is a family
of open subsets of $X$ such that $(Y \cap U_i)_{i \in I}$ is an open cover of
$Y$, we often say for simplicity that $(U_i)_{i \in I}$ is an open cover of $Y$.

\begin{dfn}
  A \emph{compact space} is a Hausdorff space $X$ such that every open cover of
  $X$ has a finite subcover.
\end{dfn}

A subset $K$ of the underlying set of a Hausdorff space $X$ is \emph{compact}
if it is a compact space when endowed with the subspace topology. It is \emph{relatively
compact in $X$} if its closure in $X$ is compact.

A compact subspace of a Hausdorff space $X$ is always a closed subset of $X$. A closed
subset of a compact space is again compact. The product of two compact spaces
is again a compact space.

\begin{thm}
  A subset $K$ of cartesian space $\set R^n$ is compact if and only if it is closed
  and there exists an $R \in \set R_+$ such that $K \subseteq \Set{p \in \set R^n : \norm p \leq R}$.
\end{thm}

The latter condition on $K$ says that $K$ is \emph{bounded}.

\section{Locally compact spaces}
\label{sec:locally_compact}

\begin{dfn}
  A Hausdorff space $X$ is \emph{locally compact} if each $p \in X$ has a
  neighborhood $U \in \mathfrak U(p, X)$, which is compact.
\end{dfn}
The condition of being locally compact for a Hausdorff space $X$ is equivalent to
requiring that every point $p \in X$ possesses an open neighborhood $U \in \mathfrak U(p, X)$,
which is relatively compact in $X$.

\begin{prop}
  \label{prop:locally_compact}
  On a locally compact Hausdorff space $X$, every neighborhood $G$ of a point $p \in X$
  contains a compact neighborhood $K \in \mathfrak U(p, X)$.
\end{prop}

The product of two locally compact Hausdorff spaces is again a locally compact
Hausdorff space.

\section{Paracompact Hausdorff spaces}
\label{sec:paracompactness}

A \emph{refinement} of an open cover $(U_i)_{i \in I}$ of a topological space $X$
is an open cover $(V_j)_{j \in J}$ of $X$ such that
\[
  \forall j \in J \, \exists i \in I: V_j \subseteq U_i.
\]

\begin{dfn}
  A \emph{paracompact Hausdorff space} is a Hausdorff space $X$ such that every
  open cover of $X$ has a locally finite refinement.
\end{dfn}

\begin{prop}
  \label{prop:paracompact_spaces}
  A paracompact Hausdorff space is normal.
\end{prop}

\begin{thm}
  The topological product of two locally compact paracompact Hausdorff spaces
  is again a (locally compact) paracompact Hausdorff space. 
\end{thm}

Cartesian space $\set R^n$ is an example of a paracompact Hausdorff space.

\section{Problems}

\begin{xca}
  Let $X$ be a topological space. Prove that a subset $U$ of the underlying set
  of $X$ is open if and only if it is a neighborhood of every point $p \in U$.
\end{xca}

\begin{xca}
  Prove that the subspace topology of a subset of the underlying set of a topological
  space is in fact a topology.
\end{xca}

\begin{xca}
  Let $Y$ be a closed subset of a topological space $X$. Let $Z$ be a subset of
  $Y$. Show that $Z$ is closed in $Y$ if and only if it is closed in $X$.
\end{xca}

\begin{xca}
  Show that \prettyref{eq:cartesian_topology} defines a topology on $\set R^n$.
\end{xca}

\begin{xca}
  Show that \prettyref{eq:product_topology} defines a topology on the cartesian
  product of the topological spaces $X$ and $Y$.
\end{xca}

\begin{xca}
  Show that the closure of a standard $\epsilon$-neighborhood in $\set R^n$
  for the canonical topology is given by
  \begin{equation}
    \overline U_{\epsilon(p)} = \Set{q \in \set R^n : \norm {q - p} \leq \epsilon}.
  \end{equation}
\end{xca}

\begin{xca}
  Let $A$ and $B$ be subspaces of the topological spaces $X$ and $Y$, respectively.
  Show that the subspace $A \times B$ of $X \times Y$ carries the
  product topology of the topological product between $A$ and $B$.
\end{xca}

\begin{xca}
  Let $X$ be a topological space. Prove that the support of a continuous map
  $\phi\colon X \to \set R$ is indeed closed and that the formula \prettyref{eq:support}
  holds.
\end{xca}

\begin{xca}
  Let $X$ and $Y$ be two topological spaces. Show that the two projection maps
  $\pr_1$ and $\pr_2$ defined on the topological product $X \times Y$ are
  continuous.
\end{xca}

\begin{xca}
  Prove that a subspace of a Hausdorff space is again a Hausdorff space.
\end{xca}

\begin{xca}
  Prove that a finite subset of a Hausdorff space is a closed subset.
\end{xca}

\begin{xca}
  Prove that a product of two Hausdorff spaces is again a Hausdorff space.
\end{xca}

\begin{xca}
  Prove that a compact subspace $K$ of a Hausdorff space $X$ is always a closed subset
  of $X$.
\end{xca}

\begin{xca}
  Prove that a closed subset $Z$ of a compact space $K$ is again compact.
\end{xca}

\begin{xca}
  Prove that the product of two compact spaces is again a compact space.
\end{xca}

\begin{xca}
  Prove that a Hausdorff space $X$ is locally compact if and only if every point
  $p \in X$ possesses an open neighborhood $U \in \mathfrak U(p, X)$ such that
  $U$ is relatively compact in $X$:
\end{xca}

\begin{xca}
  Prove that the product of two locally compact Hausdorff spaces is again a
  locally compact Hausdorff space.
\end{xca}

\begin{xca}
  Show that the canonical topology on $\set R^n$ makes cartesian space into
  a paracompact Hausdorff space.
\end{xca}

