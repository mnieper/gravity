\chapter*{Preface}

Gravity is one of the fundamental interactions of nature, by which all
physical bodies attract each other. The purpose of these notes is to
give a physically motivated and mathematically sound introduction to
Einstein's general relativity. Since its invention, it has passed all
experimental tests and is the simplest theory that is able to describe
all observed phenomena of gravity. In short, it is the accepted theory
of gravity in modern physics. This is not to say that general relativity
is set in stone. By incorporating possible intrinsic angular moment of
macroscopic  matter, one arrives at the mathematically equally beautiful
Einstein--Cartan theory, which encloses general relativity. While general
relativity is a metric theory that builds on the mathematical concept
of a Lorentzian manifold, Einstein--Cartan theory is a true gauge theory
of gravity. It will be described in the latter chapters of these notes.

Addressees of these notes are, on the one hand side, graduate students
of physics who are willing to cope with abstract mathematical concepts
and, on the other hand, graduate students of mathematics with a strong
background in classical physics. These notes grew out of a lecture
course the author gave in the winter term 2013/14 at the University of
Augsburg, and the audience of the course was a mixture of students from
the physics and the mathematical department.

The theory of general relativity and the novel effects being predicted
by it --- for example, gravitational time dilation or gravitational
waves --- have been fascinating many people, but to understand these
phenomena on a quantative level, one has to delve deeply into the
mathematics of general relativity. One the other hand, it is a much
rewarding untertaking. One will have grasped one of the most beautiful
physical theories (if not \emph{the} most beautiful one). These notes
show one of the possible routes there, a route the author would have
liked to go when he learnt general relativity. Going this route also
provides the reader with a solid knowledge of differential geometry.

That said, it \emph{is} possible to formulate the heart of the theory of
general relativity in one sentence in plain English, namely:

\begin{quote}
	The mass density measured by any observer is the scalar curvature
	of that observer's space divided by $16\pi$.
\end{quote}

Of course, without any further explanations of the contained
mathematical terms and an accompanying physical interpretation, this
statement is just as meaningful as simply stating\footnote{Gau\ss 's law
in gaussian units} $\Div E = 4 \pi \, \rho$ without any further
explanations of the terms involved. Nevertheless, the simplicity of this
statement already shows the beauty of general relativity.

\begin{center}
    Augsburg,~\today\hfill
    Marc Nieper-Wi\ss kirchen
\end{center}
