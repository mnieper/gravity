\documentclass[a4paper,12pt]{report}

\usepackage{amsmath}
\usepackage{amsthm}
\usepackage{amsfonts}
\usepackage{amsxtra}
\usepackage{mathtools}
\usepackage{upgreek}
\usepackage{prettyref}

\DeclareMathOperator{\Div}{div}
\DeclareMathOperator{\Hom}{Hom}
\DeclareMathOperator{\supp}{supp}
\newcommand{\set}[1]{\mathbf{#1}}
\newcommand{\id}{\mathrm{id}}
\newcommand{\pr}{\mathrm{pr}}
\newcommand{\diff}{\mathrm d}
\newcommand{\Tang}{\mathrm T}
\DeclarePairedDelimiter{\form}{\langle}{\rangle}
\DeclarePairedDelimiter{\Enum}{\lbrace}{\rbrace}
\DeclarePairedDelimiter{\norm}{\|}{\|}
\DeclarePairedDelimiter{\Set}{\lbrace}{\rbrace}

\theoremstyle{plain}
\newtheorem{thm}{Theorem}[chapter]
\newtheorem{lem}{Lemma}[chapter]
\newtheorem{cor}{Corollary}[chapter]
\newtheorem{prop}{Proposition}[chapter]
\theoremstyle{definition}
\newtheorem{dfn}{Definition}[chapter]
\newtheorem{xca}{Problem}[chapter]

\newrefformat{prop}{Proposition \ref{#1}}



\title{General Relativity and Einstein--Cartan Theory}
\author{Marc A. Nieper-Wi\ss kirchen}
\date{\today}

\begin{document}

  \maketitle
	
  \chapter*{Preface}

Gravity is one of the fundamental interactions of nature, by which all
physical bodies attract each other. The purpose of these notes is to
give a physically motivated and mathematically sound introduction to
Einstein's general relativity. Since its invention, it has passed all
experimental tests and is the simplest theory that is able to describe
all observed phenomena of gravity. In short, it is the accepted theory
of gravity in modern physics. This is not to say that general relativity
is set in stone. By incorporating possible intrinsic angular moment of
macroscopic  matter, one arrives at the mathematically equally beautiful
Einstein--Cartan theory, which encloses general relativity. While general
relativity is a metric theory that builds on the mathematical concept
of a Lorentzian manifold, Einstein--Cartan theory is a true gauge theory
of gravity. It will be described in the latter chapters of these notes.

Addressees of these notes are, on the one hand side, graduate students
of physics who are willing to cope with abstract mathematical concepts
and, on the other hand, graduate students of mathematics with a strong
background in classical physics. These notes grew out of a lecture
course the author gave in the winter term 2013/14 at the University of
Augsburg, and the audience of the course was a mixture of students from
the physics and the mathematical department.

The theory of general relativity and the novel effects being predicted
by it --- for example, gravitational time dilation or gravitational
waves --- have been fascinating many people, but to understand these
phenomena on a quantative level, one has to delve deeply into the
mathematics of general relativity. One the other hand, it is a much
rewarding untertaking. One will have grasped one of the most beautiful
physical theories (if not \emph{the} most beautiful one). These notes
show one of the possible routes there, a route the author would have
liked to go when he learnt general relativity. Going this route also
provides the reader with a solid knowledge of differential geometry.

That said, it \emph{is} possible to formulate the heart of the theory of
general relativity in one sentence in plain English, namely:

\begin{quote}
	The mass density measured by any observer is the scalar curvature
	of that observer's space divided by $16\pi$.
\end{quote}

Of course, without any further explanations of the contained
mathematical terms and an accompanying physical interpretation, this
statement is just as meaningful as simply stating\footnote{Gau\ss 's law
in gaussian units} $\Div E = 4 \pi \, \rho$ without any further
explanations of the terms involved. Nevertheless, the simplicity of this
statement already shows the beauty of general relativity.

\begin{center}
    Augsburg,~\today\hfill
    Marc Nieper-Wi\ss kirchen
\end{center}

  \chapter*{Special thanks}

Special thanks go to Caren Schinko for providing a written transcript of the
course on which these notes are based, and to Tim Baumann for careful
proofreading of preliminary versions of them.

  \chapter*{History}

The theory of general relativity was being developed by Albert Einstein between
1907 and 1915. It builds upon and combines the earlier theories of Newtonian
gravity and special relativity. Although the basic theory hasn't changed since
then, there have been many contributions afterwards. The following timeline
summarizes the history as far it is relevant to the notes at hand.

\begin{description}
\item[1609]
  Kepler pushlishes his first two laws of planetary motion.
\item[1619]
  Kepler pushlishes his third law of planetary motion.
\item[1638]
  Galilei's equations for a falling body.
\item[1687]
  Newton publishes his law of universal gravitation.
\item[1798]
  Cavendish measures Newton's gravitational constant.
\item[1862]
  Maxwell's equations of electromagnetism.
\item[1887]
  The Michelson-Morley experiment fails to detect a stationary
  luminiferous aether.
\item[1889]
  FitzGerald proposes Lorentz contraction.
\item[1905]
  Formulation of special relativity by Einstein.
\item[1915]
  Derivation of Einstein's field equations from an action principle by Hilbert.
\item[1915]
  Einstein's theory of general relativity.
\item[1916]
  Schwarzschild found the first exact solution of Einstein's field equations.
\item[1916]
  Einstein shows that the perihelion precession of Mercury can be fully
  explained by general relativity.
\item[1919]
  Eddington's expedition confirms that the deflection of light by the Sun is as
  predicted by general relativity.
\item[1922]
  Friedmann found a cosmological solution to Einstein's equations, in which the
  universe may expand or contract.
\item[1922]
  Introduction of the cosmological constant by Einstein into his field equations.
\item[1922]
  Proposal of the Einstein--Cartan theory by Élie Cartan.
\item[1929]
  Hubble finds evidence that the universe is expanding.
\item[1959]
  Direct measurement of the gravitational redshift of light in the Pound-Rebka
  experiment.
\item[1964]
  Discovery of the cosmis microwave background by Penzias and Wilson.
\item[1964]
  Discovery of the X-ray source Cygnus X-1, now widely accepted to be a black
  hole.
\end{description}

  \chapter*{Notations}

\paragraph{Standard sets}
We use the following notations for the standard sets: The set of natural
numbers (which includes $0$, by definition) is denoted by $\set N_0$, the set
of integers by $\set Z$, the set of rational numbers by $\set Q$, the set of
real numbers by $\set R$ and the set of complex numbers by $\set C$.

The set of the positive real numbers is denoted by a subscript: $\set R_+$. 

\paragraph{Real numbers}

For any interval $J \subseteq \set R$ and any $s \in R$, one defines the interval
\[
  - s + J \coloneqq \Set{t \in \set R : s + t \in J}.
\]

\paragraph{Linear algebra}

In an $n$-dimensional vector space, the Kronecker symbol is a scalar defined by 
\[
  \updelta_{ij} \coloneqq \begin{cases}
    1 & \text{if $i = j$} \\
    0 & \text{otherwise}
  \end{cases}
\]
for all $i$, $j \in \Set{1, \dotsc, n}$.

Given a vector space $V$, its \emph{dual space $V^*$} is the set $\Hom(V, \set R)$
of linear maps from $V$ to the scalars $\set R$, endowed with a vector space
structure by defining addition and scalar multiplication point-wise.

\paragraph{Maps}

By a \emph{differential map} we will always mean of map of class $\mathcal C^\infty$
that is a smooth map. \emph{Differentiability} thus means the existence of
continuous derivates to all orders.

The term \emph{function} will be reserved for smooth maps with values in $\set R$.
Thus a function is always a smooth function.

\paragraph{Cartesian space}
The standard basis formed by $e_1$, \dots, $e_n$ of $n$-dimensional space $\set R^n$ is
a basis of the underlying vector space such that $v = \sum_{i = 1}^n v_i \, e_i
$ for each $v = (v_1, \dotsc, v_n) \in \set R^n$. 

The standard Euclidean norm of $n$-dimensional space $\set R^n$ is denoted
by
\[
    \norm v \coloneqq \sqrt{\sum_{i = 1}^n v_i^2}
\]
for $v = (v_1, \dotsc, v_n) \in \set R^n$.

The partial derivative of a function $\phi$ defined on an open subset of $\set R^n$
in direction $i$ is denoted by $\partial_i$, that is
\[
  \partial_i \phi\colon v \mapsto \frac{\partial \phi(v + t \, e_i)}{\partial t}|_{t = 0}
\]
for $i = 1$, \dots, $n$. In case of $n = 1$, we set $\partial \coloneqq \partial_1$.

 	
  \tableofcontents
  
  \chapter{Spacetime}

\section{Introduction}

The fundamental notion of special relativity is that of an \emph{event}.
In pre-relativity, things located at a specific point in space and time,
for example the start of the Saturn V rocket launching the Apollo 11
spaceflight, are described by a unique time in Newton's absolute time and
a unique place in Euclidean absolute space. The notion of an event combines
these two qualities into one, which is essential for special relativity
as space and time by themselves lose their absoluteness.

\emph{Spacetime} is the set of all possible events. Any event that can
be imagined to happen is an element or a
\emph{point} in spacetime. Mathematically, spacetime has more
structure than simply being a set: A point in this set, that is an event, is
usually described by four scalars, one time and three space coordinates,
making spacetime four-dimensional. For a general set, however, there is no
well-defined notion of coordinates or dimension. The theory of manifolds, which
will be presented in the next section, is the correct mathematical setting in
which notions like coordinates and dimension make sense.

\section{Manifolds}
\label{sec:manifolds}

An \emph{$n$-dimensional chart $x = (x, U)$} on a set $M$, whose elements we call
\emph{points}, is an injective map
$x = (x_1, \dotsc, x_n)\colon U \to \set R^n$ onto an open subset of $\set R^n$ defined on a subset
$U$ of $M$, the \emph{domain of definition of $x$}. Given a point $p$ lying in the
domain of definition $U$, the $n$ scalars $x_1(p)$, \dots, $x_n(p)$ are the
\emph{coordinates of $p$ with respect to the chart $x$}. A
family $\Set{(x_i, U_i)}_{i \in I}$ of charts on $M$ \emph{covers} $M$ if $M \subseteq
\bigcup_{i \in I} U_i$ that is if every point $p$ of $M$ lies in at least domain
of definition of the charts $x_i$. In order to be able to describe events in
spacetime by four coordinates, we postulate that \emph{there is a distinguished
family of four-dimensional charts that cover spacetime}.

Given two $n$-dimensional charts $(x, U)$ and $(y, V)$ of a set $M$, the map
\[
  y \circ x^{-1}|x(U \cap V)\colon x(U \cap V) \to y(U \cap V)
\]
is called the
\emph{coordinate transformation from $x$ to $y$}. If this map is a diffeomorphism
between open subsets of $\set R^n$, the charts $x$ and $y$ are said to be
\emph{compatible}.
An \emph{$n$-dimensional atlas of $M$} is a family of pairwise compatible
$n$-dimensional charts of $M$ that covers $M$. In order to be able to employ 
analytic methods, we extend our postulate above by postulating that
\emph{there is a distinguished four-dimensional atlas of spacetime}.

Let $\mathfrak A$ be any $n$-dimensional atlas of a set $M$, and let $x$ and $y$
be two arbitrary $n$-dimensional charts of $M$. If $x$ and $y$ are each compatible
with each chart in $\mathfrak A$, they are compatible with each other. Therefore,
every atlas $\mathfrak A$ can be uniquely enlarged to a \emph{maximal atlas} in
which it is contained by adding all $n$-dimensional charts to $\mathfrak A$ that
are compatible with each in chart in $\mathfrak A$. We also say that the maximal
atlas is \emph{generated} by the charts contained in $\mathfrak A$.

An \emph{$n$-dimensional premanifold $M$} is set $M$ together
with a maximal $n$-dimensional atlas $\mathfrak U^\infty(M)$ of $M$. The set of
all charts $(x, U)$ in $\mathfrak U^\infty(M)$ that contains a given point $p \in M$ is 
denoted by $\mathfrak U^\infty(p, M)$. A \emph{chart of $M$} is a chart in the
maximal atlas $\mathfrak U^\infty(M)$.
An \emph{atlas of the premanifold $M$} is any atlas of $M$ which is contained in
the maximal atlas $\mathfrak U^\infty(M)$. With these terms, we can say that
\emph{spacetime is a four-dimensional premanifold}.

Using the atlas of a premanifold $M$, one can define the notion of neighborhoods
of points on $M$: A subset $G$ of the underlying set of $M$ is an \emph{open subset
of $M$} if $x(U \cap G)$ is an open subset for each chart $(x, U)$ of $M$. The
system of these open subsets of $M$ is a topology (see~\prettyref{sec:topological_spaces})
on the underlying set of $M$, the \emph{canonical topology of $M$}, so that $M$
is a topological space in a canonical way.

For any chart $(x, U)$ of the $n$-dimensional premanifold, a subset $V$ of $U$
is open in $M$ if and only if $x(V)$ is open in $\set R^n$. In particular, $U$ is open.
The map $x\colon U \to \set R^n$ is continuous in the sense of maps between
topological spaces.

Generally, the so-defined topology is ill-behaved, however;
points may not be distinguishable by the topology and one may need infinitely
many charts to cover even small neighborhoods of points. Excluding these cases
by adding technical conditions on the underlying topological space leads to the
final definition:
\begin{dfn}
  An \emph{$n$-dimensional manifold $M$} is an $n$-dimensional premanifold $M$
  whose underlying topological space is a paracompact Hausdorff space
  (see~\prettyref{sec:paracompactness}).
\end{dfn}
We extend our postulate from above by postulating that \emph{spacetime is a
four-dimensional manifold}.

The most basic example of an $n$-dimensional manifold is the $n$-dimensional 
cartesian space $\set R^n$. Its canonical atlas is the maximal atlas which is generated
by the identity map $\id_{\set R^n}\colon \set R^n \to \set R^n$ viewed
as an $n$-dimensional chart. The topology defined by this atlas is, of course,
the canonical topology of $\set R^n$, which is the topology of a paracompact
Hausdorff space.

Requiring that the underlying topological space of spacetime is a paracompact
Hausdorff space has a number of pleasant consequences: For example, the underlying
topological spaces of manifolds are normal, which follows from \prettyref{prop:paracompact_spaces}.
Furthermore, we have:
\begin{prop}
  The underlying space of an $n$-dimensional manifold $M$ is locally compact.
\end{prop}
(For the definition of local compactness, see \prettyref{sec:locally_compact}.)

\begin{proof}
  For $p \in M$ choose a chart $(x, U) \in \mathfrak U^\infty(p, M)$. By openness
  of $x(U)$ in $\set R^n$, there exists an $\epsilon > 0$ such that $U_\epsilon(x(p))
  \subseteq x(U)$. We claim that
  \[
    K \coloneqq x^{-1}(\overline{U_{\frac \epsilon 2}(x(p))})
  \]
  is a compact neighborhood of $p$:
  
  The subset $K$ is a neighborhood of $p$ as
  $p \in x^{-1}(U_{\frac \epsilon 2}(x(p))) \subseteq K$ and
  because $x^{-1}(U_{\frac 1 2 \epsilon}(x(p)))$ is an open subset of $M$.

  To show that $K$ is compact, let $(U_i)_{i \in I}$ be a family of open subsets
  of $U$ that cover $K$. By definition of the topology of $M$, the images $(x(U_i))_{i \in I}$
  form an open cover of the compact space $\overline{U_{\frac \epsilon 2}(x(p))}$.
  Thus there is a finite subset $J \subset I$ such that $(x(U_i))_{i \in J}$ is
  an open cover of $\overline{U_{\frac \epsilon 2}(x(p)}$. It follows that
  $(U_i)_{i \in J}$ is a finite open cover of $K$.
\end{proof}

Often, we have to restrict our attention to small pieces (that is, open subsets)
of a given $n$-dimensional manifold $M$, for example spacetime. This can be
done as follows: Let $G$ be an open subspace of $M$. Let $\mathfrak U^\infty(G)$
be the unique maximal $n$-dimensional atlas of $G$ that contains the atlas
\[
  \Set{(x|U \cap G, U \cap G) : (x, U) \in \mathfrak U^\infty(M)}.
\]
Then $G$ becomes an $n$-dimensional manifold itself, whose underlying topological
space is the subspace $G$ of $M$. An $n$-dimensional manifold of this form is
called an \emph{open submanifold of $M$}. The domains of definitions of the charts of
$M$ are open submanifolds of $M$.

\section{Functions on a manifold}
\label{sec:functions}

As we have postulated, we can ``measure'' each event in spacetime by giving the
scalar values of four coordinates (after choosing a chart). Each coordinate can
be thought of as a scalar field that assigns to each point (in the domain of
definition of its chart) a scalar value. This notion is generalised as follows:

\begin{dfn}
  A \emph{(smooth) function $\phi\colon M \to \set R$} on an $n$-dimensional
  manifold $M$ is a mapping $\phi$ from the underlying set of $M$ to the reals
  such that for each chart $(x, U)$ of $M$ the map
  \[
    \phi \circ x^{-1}\colon x(U) \to \set R
  \]
  is smooth.
\end{dfn}

In order to show that a map $\phi\colon M \to \set R$ is a function in the above
sense, it suffices that
for each point $p \in M$ there exists a chart $(x, U) \in \mathfrak U^\infty(p, M)$
such that $\phi \circ x^{-1}\colon x(U) \to \set R$ is smooth.

Every function is continuous with respect to the underlying
topologies of $M$ and $\set R$, respectively. Let $\phi$ be a function on $M$
and $G$ an open submanifold. The \emph{restriction $\phi|G$ of $\phi$ to $G$},
given by
\[
  \phi|G\colon G \to M, p \mapsto f(p)
\]
is a function on the manifold $G$. For any chart $(x, U)$ of $M$, the coordinate
functions $x_1$, \dots, $x_n\colon U \to M$ are smooth functions on the open
submanifold $U$.

All functions on $M$ form an algebra (over the reals), denoted by
$\mathcal C^\infty(M)$, where addition and multiplication are defined point-wise.
The function with constant value $c \in \set R$ is often denoted by $\underline c$.
The sets of functions of the open submanifolds of $M$ fulfill the \emph{sheaf
condition}, that is for every open cover $(U_i)_{i \in I}$ of $M$ and functions
$\phi_i \in \mathcal C^\infty(U_i), i \in I$ one has
\[
  \left(\forall i, j \in I : \phi_i|U_i \cap U_j = \phi_j|U_i \cap U_j\right)
  \implies \exists! \phi \in \mathcal C^\infty(M) \, \forall i \in I :
  \phi|U_i = \phi_i.
\]
In other words, we can uniquely glue functions along open submanifolds.

If $\Phi\colon \set R \to \set R$ is any smooth function between the reals, 
the composition $\Phi \circ \phi\colon M \to \set R$ is a function on $M$
whenever $\phi\colon M \to \set R$ is a function.

A manifold possesses many functions in a sense made precise by the following theorem,
which relies essentially on the paracompactness of the manifold (for the notion
of the support $\supp$ of a function, see \prettyref{sec:continuity}):
\begin{thm}
  \label{thm:paracompact_manifold}
  Every open cover $(U_i)_{i \in I}$ of a manifold $M$ has a
  \emph{subordinate
  partition $(\lambda_i)_{i \in I}$ of unity}, which is a family of functions
  $\lambda_i\colon M \to \set R$ with the following properties:

  \paragraph{Range}
  For all $i \in I$ and $p \in M$, one has $0 \leq \lambda_i(p)
  \leq 1$.
  
  \paragraph{Support}
  For all $i \in I$, one has $\supp \lambda_i \subseteq U_i$.
  
  \paragraph{Local finiteness}
  Every $p \in M$ posesses a neighborhood $G \in \mathfrak U^0(p, M)$ such that there
  are only finitely many $i \in I$ with $\supp \lambda_i \cap G \neq \emptyset$.
  
  \paragraph{Normalization}
  For every $p \in M$, the equality $\sum_{i \in I} \lambda_i(p) \equiv 1$ holds.
\end{thm}

The proof relies on the following lemma, which is also of independent interest:
\begin{lem}
  \label{lem:paracompact_manifold}
  Let $K$ be a compact subspace of an $n$-dimensional manifold $M$. For any open neighborhood
  $G$ of $K$ in $M$, there exists a function $\phi\colon M \to \set R$ with
  \begin{align}
    \label{eq:paracompact_manifold}
    \forall p \in M: \phi(p) & \ge 0, &
    \forall p \in K: \phi(p) & > 0, &
    \supp \phi & \subseteq G.
  \end{align}
\end{lem}

\begin{proof}[Proof of \prettyref{lem:paracompact_manifold}]
  For any $p \in K$ choose a chart $(x, U) \in \mathfrak U^\infty(p, M)$. 
  By the openness of $x(U)$ in $\set R^n$, there exists an $\epsilon > 0$ with
  $U_\epsilon(x(p)) \subseteq x(U \cap G)$. Choose a bump function $\psi\colon \set R^n \to
  \set R$ (see \prettyref{sec:bump_functions}) such that $\psi(u) \ge 0$ for all
  $u \in \set R^n$, $\psi(u) = 1$ for
  $\norm u \leq \frac \epsilon 3$ and $\psi(u) = 0$ for $u \ge \frac {2 \epsilon} 3$.
  By the sheaf condition, 
  \[
    \phi_p\colon M \to \set R, q \mapsto \begin{cases}
      \psi(x(q) - x(p)) & \text{if $q \in U$}, \\
      0 & \text{if $q \in M \setminus x^{-1}(\overline{U_{\frac {2 \epsilon} 3}(x(p))})$}
    \end{cases}
  \]
  defines a function on $M$ as $U$ and $M \setminus x^{-1}(\overline{U_{\frac {2 \epsilon} 3}(x(p))})$
  form an open cover of $M$. This function has the properties
  \begin{align*}
    \forall q \in M: \phi_p(q) & \ge 0, &
    \phi_p(p) & > 0, &
    \supp \phi_p & \subseteq G.
  \end{align*}
  
  The open subsets $U_p \coloneqq \Set{q \in M : \phi_p(q) > 0}$ with $p \in K$ cover $K$. By
  compactness of $K$, there exists a finite subset $A \subseteq K$ such that
  $K$ is covered by $(U_p)_{p \in K}$. By construction, the function
  $\phi = \sum_{p \in A} \phi_p$
  fulfills \prettyref{eq:paracompact_manifold}.
\end{proof}

\begin{proof}[Proof of \prettyref{thm:paracompact_manifold}]
  For every $p \in M$ choose by local compactness of $M$ a relatively compact neighborhood $G_p
  \in \mathfrak U(p, M)$. By the covering property, there exists an $i \in I$ with
  $p \in U_i$. The intersection $G_p \cap U_i$ is again relatively compact in $M$,
  so we may assume that already $G_p \subseteq U_i$. So  
  $(G_p)_{p \in M}$ is a refinement of the cover $(U_i)_{i \in I}$.
  Let $(V_j)_{j \in J}$ be a locally finite refinement of $(G_p)_{p \in M}$. In
  particular, $(V_j)_{j \in J}$ is a locally finite refinement of $(U_i)_{i \in I}$
  and each $V_j$ is relatively compact in $M$. By the shrinking lemma, \prettyref{prop:shrinking_lemma},
  and the normality of the underlying topological space of $M$, there exists
  another open cover $(V'_j)_{j \in J}$ of $M$ with
  $\overline{V'_j} \subseteq V_j$ for all $j \in J$.
  
  For every $j \in J$, choose by compactness of $\overline{V'_j}$ and
  \prettyref{lem:paracompact_manifold} a function
  $\phi_j \in \mathcal C^\infty(M)$ with
  \begin{align*}
    \forall p \in M: \phi_j(p) & \ge 0, &
    \forall p \in V'_j: \phi_j(p) & > 0, &
    \supp \phi_j \subseteq V_j.
  \end{align*}
  By the local finiteness of the open cover $(V_j)_{j \in J}$, the sum
  $\phi \coloneqq \sum_{j \in J} \phi_j$ is locally a finite sum and thus defines
  a function on $M$ by the sheaf condition. By the covering property of
  $(V'_j)_{j \in J}$,
  one has $f(p) > 0$ for all $p \in M$.
  
  As $(V_j)_{j \in J}$ is a refinement of the cover $(U_i)_{i \in I}$, there exists a
  map $\alpha\colon J \to I$ with $V_j \subseteq U_{\alpha(i)}$ for all $j \in J$.
  For each $i \in I$ set $J_i \coloneqq \alpha^{-1}(i)$, so $(J_i)_{i \in I}$ becomes
  a partition of $I$. For all $i \in I$, set
  $U'_i \coloneqq \bigcup_{j \in J_i} V_j \subseteq U_i$ and finally
  \[
    \lambda_i = \sum_{j \in J_i} \frac{\phi_i}{\phi}.
  \]
  By a similar argument as above, the sum on the right hand side is locally finite
  and, thus, $\lambda_i$ is a function on $M$ with $\lambda_i(p) \ge 0$ for all
  $p \in M$.
  
  By construction, $\supp \lambda_i \subseteq \bigcup_{j \in J_i} V_j = U'_j \subseteq
  U_i$, which proves the support axiom of a partition of unity. The covering
  $(U'_i)_{i \in I}$ is locally finite as the covering $(V_j)_{j \in J}$ is locally
  finite; this uses the disjointness of the $J_i$. From $\supp \lambda_i \subseteq U'_i$
  for all $i \in I$, the local finiteness axiom follows. By construction,
  $\sum_{i \in I} \lambda_i \equiv 1$, which is the normalization axiom of a
  partition of unity.
  From this, the range axiom follows as we already know that $\lambda_i(p) \ge 0$
  for all $p \in M$.
\end{proof}

The existence of partitions of unity on manifolds implies that functions can
be extended in the following sense:

\begin{cor}
  Let $\phi$ be a function defined on an open
  neighborhood $G$ of a point $p$
  in a manifold $M$. Then there exists a function $\widehat \phi \in \mathcal C^\infty(M)$
  with $\supp \widehat\phi \subseteq G$ and
  such that $\widehat \phi$ coincides with $\phi$ on a neighboorhood $U \subseteq G$
  of $p$ in $M$.
\end{cor}
The function $\widehat \phi$ is called an \emph{extension of $\phi$
by zero away from $p$}.

\begin{proof}
  By local compactness (see \prettyref{prop:locally_compact}),
  there exists a compact neighborhood $K \in \mathfrak U(p, G)$.
  Choose a partition $(\lambda, \mu)$ of unity subordinate to the open cover
  $(G, M \setminus K)$ of $M$. By the sheaf condition,
  \[
    \hat\phi\colon M \to \set R,
    \begin{cases}
      \lambda(p) \cdot \phi(p) & \text{if $p \in G$} \\
      0 & \text{if $p \in M \setminus \supp \lambda$}
    \end{cases}
  \]
  is a well-defined function on $M$ with $\supp \hat\phi \subseteq G$ and
  which coincides with $\phi$ on $K$.
\end{proof}

For any two points $p$, $q \in M$ with $p \neq q$, there exists an open
neighborhoods $G \in \mathfrak U^0(p, M)$ with $q \notin G$. Extending the
constant function $\underline 1|G$ by zero away from $p$ yields a function,
which is $1$ on $p$ and $0$ on $q$. Thus, we have
\begin{equation}
  \label{eq:point_separation}
  \forall p, q \in M : p \neq q \implies
  \exists \lambda \in \mathcal C^\infty(M) : \lambda(p) = 1, \lambda(q) = 0.
\end{equation}

If we denote by $p^*$ for all $p \in M$ the algebra homomorphism
\begin{equation}
  p^*\colon \mathcal C^\infty(M) \to \set R, \phi \mapsto \phi(p),
\end{equation}
we can reformulate \prettyref{eq:point_separation} by saying that the map $p \mapsto p^*$ is injective,
that is the algebra of functions \emph{separate points}. The algebra of functions
on spacetime is therefore a full set of observables: for any two distinct events
there is a function that takes different values on both events.

\section{Morphisms}
\label{sec:morphisms}

A physical body traces out a curve of events in spacetime $M$, namely those events
where an observer meets the physical body, its \emph{world line}. If the physical
body carries a clock with it, each point of its world line is parametrized by
a scalar, the clock's time measured at that event. In other words, the path of
the physical body in spacetime together with its clock defines a map $J \to M$, 
where $J$ is an (open) interval. Both the domain and the target of this map is a 
manifold, where $J$ is viewed as an open submanifold of the reals. A physical
body does not jump through spacetime, so the map will be continuous.
In order to employ analytical methods, it is sensible to assume moreover that
this map is a morphism according to the following definition:

\begin{dfn}
  A \emph{morphism $f\colon M \to N$} between two manifolds $M$ and $N$ is a
  continuous map $f\colon M \to N$ such that for each pair of charts
  $(x, U) \in \mathfrak U^\infty(M)$ and $(y, V) \in \mathfrak U^\infty(N)$ the
  composition
  \[
    y \circ f \circ x^{-1}|x(f^{-1}(V) \cap U)\colon x(f^{-1}(V) \cap U) \to y(V) 
  \]
  is a smooth map (between open subsets of cartesian spaces).
\end{dfn}

In accordance with our above wording, a \emph{curve $\alpha\colon J \to M$} is a
morphism where $J$ is an open interval viewed as a submanifold of $\set R$. Thus
a world line of a physical body becomes a curve in this sense
by endowing it with a clock.

A map $f\colon M \to N$ between manifolds is a morphism if and only if
for all functions $\psi \in \mathcal C^\infty(N)$ the \emph{pullback of $\psi$
by $f$}, given by
\begin{equation}
  f^{-1}\psi\colon M \to \set R, p \mapsto \psi(f(p)),
\end{equation}
is a function on $M$. (The pullback itself is a map
\begin{equation}
  f^{-1}\colon \mathcal C^\infty(N) \to \mathcal C^\infty(M)
\end{equation}
of algebras.)

Thus, a map $\phi\colon M \to \set R$ defined on a manifold $M$ is a
morphism if and only if it is a function. Further, a map $f\colon G \to H$ between
open subsets of cartesian spaces is a morphism if and only if it is a smooth
map in the sense of calculus.

The identity $\id_M\colon M \to M$ of $M$ is a morphism. The composition $g \circ f$
of two morphisms $f\colon M \to N$ and $g\colon N \to P$ between manifolds is
again a morphism. The manifolds together with the morphisms between them thus
form a \emph{category}.

The \emph{inclusion $i\colon U \to M, p \mapsto p$} of an open submanifold $U$
of $M$ is a morphism. Thus, the restriction $f|U = f \circ i$ of a morphism
$f\colon M \to N$ to $U$ is again a morphism.

\section{Vectors}
\label{sec:vectors}

By an \emph{observer}, we mean a curve $\alpha\colon J \to M$ in spacetime $M$,
which might be traced out by a physical body, and relative to which one may
measure events. First and foremost, an observer defines a set of events, namely
the set $\alpha(J)$ of events traced out by the observer. This defines the
relation of being at the same spatial location as the observer: an event $p$ is 
\emph{at the same spatial location as the oberver} if and only if
$p \in \alpha(J)$. The curve $\alpha$, however, is not determined by the set
$\alpha(J)$ of events alone; it does not capture the dynamics of $\alpha$.
For example, if we move infinitesimally from $t$ to $t + \diff t$ on $J$, we
expect that $\alpha(t)$ moves infinitesimally to $\alpha(t) + d \alpha(t)$. In
other words, we expect $\frac{\diff\alpha(t)}{\diff t}$ to be a vector on $M$
tangent to the curve $\alpha(t)$. At the moment, howoever, we have no notion of
such a thing as a vector, which with we could describe dynamics, on a manifold $M$.
The purpose of this section is to remedy this.

The basic idea of defining vectors on a manifold $M$ is the following:
Any point $p$ assigns a value $p^* \phi = \phi(p)$ to each function
$\phi \in \mathcal C^\infty(M)$, and we can reconstruct the point from these
values as the map $p \mapsto p^*$
is injective. Assume that we already have the notion of a vector $v$ at a point $p$.
It would assign a change of values along $v$, a derivative, written $v \cdot \phi$, to each
function $f$. Thus such a vector would give rise to a function
\[
  v\colon \mathcal C^\infty(M) \to \set R, \phi \mapsto v \cdot \phi.
\]
A vector should be reconstructable from this map, so that the functions form
a complete set of observables for vectors. This map should have the
properties of a first-order derivative operator, so we end up with the following
definition:

\begin{dfn}
  A \emph{vector $v$ at a point $p \in M$} is a map
  \[
    v\colon \mathcal C^\infty(M) \to \set R, \phi \mapsto v \cdot \phi,
  \]
  such that the following axioms hold:
  
  \paragraph{Constants}
  For all $c \in \set R$, one has $v \cdot \underline c = 0$.

  \paragraph{Linearity}
  For all $a$, $b \in \set R$ and all $\phi$, $\psi \in \mathcal C^\infty(M)$,
  one has $v \cdot (a \, \phi + b \, \psi) = a \, v \cdot \phi + b \, v \cdot \psi$.
  
  \paragraph{Leibniz rule}
  For all $\phi$, $\psi \in \mathcal C^\infty(M)$, one has
  $v \cdot (\phi \, \psi) = (v \cdot \phi) \, \psi(p) + \phi(p) \, (v \cdot \psi)$.
\end{dfn} 

The real number $v \cdot \phi$ is the \emph{derivative of $\phi$ along $v$}. For
example, the constants axiom says that constants have vanishing derivative along
any vector.

The set of all vectors at a point $p \in M$ is called the \emph{tangent space
of $M$ at $p$} and is denoted by $\Tang_p M$. The tangent space $\Tang_p M$ is
canonically a (real) vector space, where addition and scalar multiplication
are defined ``function-wise'':
\begin{align*}
  0 \cdot \phi & \coloneqq 0,&
  (a \, v) \cdot \phi & \coloneqq a \, (v \cdot \phi), &
  (v + w) \cdot \phi & \coloneqq v \cdot \phi + w \cdot \phi
\end{align*}
for all $v$, $w \in \Tang_p M$, $a \in \set R$ and $\phi \in \mathcal C^\infty(M)$.

The first basic result, which partially shows that the above definition of a 
vector in fact coincides with the intuitive notion, is that the derivative of
a function $\phi$ along a vector at a point $p$ depends only on the values of $\phi$
in a neighborhood of $p$:

\begin{lem}
  Let $G$ be an open neighborhood of a point $p \in M$ such that $\phi|G = \psi|G$
  for two functions $\phi$ and $\psi \in \mathcal C^\infty(M)$. Then
  \[
    \forall v \in \Tang_p M: v \cdot \phi = v \cdot \psi.
  \]
\end{lem}

\begin{proof}
  By \prettyref{lem:paracompact_manifold}, choose a smooth function $\lambda \in
  \mathcal C^\infty$ with $\lambda(p) = 1$ and $\supp \lambda \subseteq G$. Then
  \[
    \begin{split}
      v \cdot \phi - v \cdot \psi
      & = v \cdot (\phi - \psi) = v \cdot \left((1 - \lambda) \, (\phi - \psi)\right) \\
      & = \left(v \cdot (1 - \lambda)\right) \, (\phi(p) - \psi(p))
      + (1 - \lambda(p)) \, \left(v \cdot (\phi - \psi)\right) = 0 + 0 \\
      & = 0.
    \end{split}
  \]
\end{proof}

Given a vector $v$ at $p$ and a function $\phi$ defined in a neighborhood $G$
of $p$, we can therefore speak of the derivative $v \cdot \phi$ of $\phi$ along
$v$: Let $\widehat\phi$ be any extension of $\phi$ by zero away from $p$. As
$\widehat\phi$ and $\phi$ coincide in a neighborhood of $p$, the derivative
\[
  v \cdot \phi \coloneqq v \cdot \widehat\phi
\]
does not depend on the choice of $\widehat\phi$.

With this result we can completely determine the structure of the tangent space
$\Tang_p M$. In fact, we have:

\begin{thm}
  \label{thm:coordinate_basis}
  Let $M$ be an $n$-dimensional manifold. For each point $p \in M$, the tangent
  space $\Tang_p M$ is an $n$-dimensional vector space. More precisely, given
  a chart $(x, U) \in \mathfrak U^\infty(p, M)$, a basis of $\Tang_p M$ is given
  by the vectors
  \begin{equation}
    \label{eq:coordinate_basis}
    \frac{\partial}{\partial x_i}|_p\colon \mathcal C^\infty(M)
    \to \set R, f \mapsto \delta_i ((f|U) \circ x^{-1}) (x(p))
    \eqqcolon \frac{\partial f}{\partial x_i}(p),  
  \end{equation}
  where $i = 1, \dotsc, n$. If $(y, V) \in \mathfrak U^\infty(p, M)$ is another
  chart around $p$, the change of basis is given by
  \begin{equation}
    \label{eq:basis_change}
    \frac{\partial}{\partial x_i}|_p =
    \sum_{j = 1}^n \partial_i ((y_j|U) \circ x^{-1}) (x(p))
    \, \frac{\partial}{\partial y_j}|_p
    =
    \sum_{j = 1}^n \frac{\partial y_j}{\partial x_i}(p) \, 
    \frac{\partial}{\partial y_j}|_p.
  \end{equation}
\end{thm}
The basis $(\frac{\partial}{\partial x_i}|_p)_i$ is called the
\emph{chart's $x$ induced basis of $\Tang_p M$}.

\begin{proof}
  To show that \prettyref{eq:coordinate_basis} defines a vector at $p$ is
  straight-forward and follows from the linearity and product rule of the
  partial differentiation in calculus.
  
  The linear independence of the $n$ vectors defined by
  \prettyref{eq:coordinate_basis} follows directly from the following
  observation: For any $n$-tupel $(a_1, \dotsc, a_n)$ of
  scalars with $\sum_{i = 1}^n a_i \, \frac\partial{\partial x_i}|_p = 0$, one
  calculates
  \[
    \begin{split}
      0 & = \left(\sum_{i = 1}^n a_i \, \frac\partial{\partial x_i}|_p\right) \cdot x_j
      = \sum_{i = 1}^n a_i \, \partial_i (x_j \circ x^{-1}) (x(p)) \\
      & = \sum_{i = 1}^n a_i \, \updelta_{ij}
      = a_j
    \end{split}
  \]
  for all $i = 1$, \dots, $n$.
  
  To show that the $n$ vectors defined by \prettyref{eq:coordinate_basis} span
  the tangent space $\mathrm T_p M$ at $p$, let $\phi$ be any function on $M$. By
  \prettyref{thm:hadamard}, there exist functions $g_1$, \dots, $g_n$ defined
  in a neighborhood of $p$ in $U$ such that
  \[
    \phi = \phi(p) + \sum_{i = 1}^n (x_i - x_i(p)) \, g_i(u)
  \]
  in a neighborhood of $p$ in $\set R^n$. Given any vector $v \in
  \mathrm T_p M$, we thus have
  \[
    \begin{split}
      v \cdot \phi & = v \cdot (\phi(p) + \sum_{i = 1}^n (x_i - x_i(p)) \, g_i) \\
      & = \sum_{i = 1}^n \left((v \cdot (x_i - x_i(p))) \, g_i(p) + (x_i(p) - x_i(p)) \, (v \cdot g_i)\right)
      = \sum_{i = 1}^n (v \cdot x_i) \, g_i(p).
    \end{split}
  \]
  In other words, $v \cdot \phi = \sum_{j = 1}^n (v \cdot x_j) \, \frac\partial{\partial x_j}|_p \cdot \phi$
  for any vector $v \in \mathrm T_p M$. Thus
  \[
    \forall v \in \mathrm T_p M: v = \sum_{i = 1}^n (v \cdot x_j) \, \frac\partial{\partial x_i}|_p,
  \] 
  which is a linear combination of the induced basis.
  
  As this formula also works for the coordinate system $(y, V)$, we get
  \prettyref{eq:basis_change} for $v = \frac\partial{\partial x_i}|_p$.
\end{proof}


% TODO Pushforward of vectors;
% TODO Show that T_p U = T_p M by locality.

\section{Tensors}

\section{Vector fields}

\section{Tensor fields}

\section{Problems}

\begin{xca}
  Prove that every $n$-dimensional atlas of a set $M$ is contained in a unique
  maximal $n$-dimensional atlas of $M$.
\end{xca}

\begin{xca}
  Prove that the system of open subsets of a premanifold $M$ is a topology on
  the underlying set of $M$.
\end{xca}

\begin{xca}
  Let $(x, U)$ be a chart of an $n$-dimensional premanifold $M$. Show that a 
  subset $V$ of $U$ is open in $M$ if and only if $x(V)$ is open in $\set R^n$.
  Conclude that $x\colon U \to \set R^n$ is a continuous map.
\end{xca}

\begin{xca}
  Prove that an open submanifold of a manifold is in fact a manifold.
\end{xca}

\begin{xca}
  Prove the following:
  Let $M$ be a manifold and $\phi\colon M \to \set R$ a map. Assume that for each
  point $p \in M$ there exists a chart $(x, U) \in \mathfrak U^\infty(p, M)$
  such that $\phi \circ x^{-1}\colon x(U) \to \set R$ is smooth. Then $\phi$
  is a (smooth) function.
\end{xca}

\begin{xca}
  Prove that every function $f$ on an $n$-dimensional manifold $M$ is a
  continuous map $f\colon M \to \set R$ for the underlying topologies of $M$ and
  $\set R$.
\end{xca}

\begin{xca}
  Prove that the functions on a manifold $M$ form a \emph{sheaf of algebras},
  that is prove that $\mathcal C^\infty(U)$ is closed under constants,
  addition and multiplication for every open subset $U$ of $M$, that the restrictions
  $\mathcal C^\infty(U) \to \mathcal C^\infty(V)$ are homomorphisms of algebras
  for every inclusion $V \subseteq U$ of open subsets of $M$, and that 
  for every open cover $(U_i)_{i \in I}$ of $M$ and functions
  $\phi_i \in \mathcal C^\infty(M)$ one has
  \[
    \left(\forall i, j \in I : \phi_i|U_i \cap U_j = \phi_j|U_i \cap U_j\right)
    \implies \exists! \phi \in \mathcal C^\infty(M) \, \forall i \in I :
    \phi|U_i = \phi_i.
  \]
\end{xca}

\begin{xca}
  Prove that a map $f\colon M \to N$ between manifolds is a morphism if and only if
  \[
    \forall \psi \in \mathcal C^\infty(M):
    f^{-1} \psi \in \mathcal C^\infty(N).
  \]
\end{xca}

\begin{xca}
  Prove that the composition $g \circ f$ or two morphisms $f\colon M \to N$
  and $g\colon N \to P$ between manifolds is again a morphism.
\end{xca}

\begin{xca}
  Proof that the inclusion morphism $i\colon U \to M$ for any open submanifold
  $U$ of a manifold $M$ is in fact a morphism.
\end{xca}


  \include{infinitesimal}
  
  \chapter{Equations of motion in spacetime}

\section{Introduction}

\section{Affine manifolds}

\section{Parallel displacement}

\section{Torsion}

\section{Autoparallels}

\section{The exponential map}

\section{Problems}
 % Equations of motions
  % Introduction
  % Affine manifolds, connection coefficients
  % Constant tensor fields & parallel displacement
  % Second derivative & torsion
  % Autoparallel
  % Exponential map, normal coordinates
  
  \chapter{Tidal forces}

\section{Introduction}

\section{Curvature}

\section{Bianchi identities}

\section{Jacobi fields}

\section{Taylor expansion of the exponential map}

\section{Problems}

 % Tidal forces
  % Introduction
  % Riemann curvature
  % Jacobi fields
  % Bianchi identities
  % Exponential map
  
  \chapter{Vacuum structure and causality}

\section{Introduction}

\section{Pseudo-riemannian manifolds}

\section{Levi--Civita connection}

\section{Geodesics}

\section{Riemannian curvature}

\section{Geodesic discs and balls}

\section{Ricci curvature}

\section{Problems}
 % 
  % Introduction
  % Riemannian and Pseudo-riemannian manifolds
  % Levi-Civita connection
  % Geodesics
  % Riemannian curvature
  % Interpretation of Riemannian curvature 
  % Interpretation of scalar curvature
  % Interpretation of Ricci curvature in Lorentzian manifolds
  
  \chapter{General relativity}

\section{Introduction}

\section{Einstein tensor}

\section{Mass density}

\section{Einstein's field equations}

\section{Newton's theory of gravity}

\section{Equations of motion}

\section{Killing vector fields}

\section{Schwarzschild metric}

\section{Schwarzschild potential}

\section{Problems}
 % General relativity
  % Introduction
  % Einstein tensor
  % Mass density
  % Einstein's field equations
  % Newton's theory of gravity
  % Equations of motion
  % Killing vector fields
  % Schwarzschild metric
  % Schwarzschild potential
  
  \chapter{Electrodynamics}

\section{Differential forms}

% vector potential

% 

  % Introduction
  % Differential forms
  % Hodge dual
  % Maxwell theory
  % energy momentum tensor of electrodynamic
  % charged black hole
  
  \chapter{Action principles}

\section{Introduction}

\section{Euler-Lagrange equation}

\section{Conserved currents}

\section{Maxwell's theory}

\section{Hilbert's action principle}

\section{Problems}

 % Action principle
  % Introduction
  % Euler-Lagrange eqn. for fields
  % Currents
  % Maxwell theory
  % Relativity
  
  \chapter{Einstein--Cartan theory}

\section{Introduction}

\section{Lie groups}

\section{Fibre bundles}

\section{Ehresmann connections}

\section{Electromagnetism}

\section{Cartan connections}

\section{Einstein--Cartan theory}

\section{Spin}

\section{Equations of motion}

\section{Problems}

 % Gauge theories
  % Introduction
  % Fibre bundles
  % Ehresmann connections
  % U(1) theory
  % Cartan connections
  % SO(1,4) theory
  % eq. of motion with torsion
  % spin/fermions
  
	\appendix
	
  \chapter{Topology}

\section{Topological spaces}
\label{sec:topological_spaces}

\begin{dfn}
  Let $X$ be a set. A \emph{topology on $X$} is a subset of the
  power set of $X$, whose elements are called \emph{open subsets of the topology}
  and which fulfills the following axioms:
  
  \paragraph{Union}
  Let $(U_i)_{i \in I}$ be a family of open subsets of the topology. Then the
  union $\bigcup_{i \in I} U_i$ is again an open subset of the topology.
  
  \paragraph{Intersection}
  Let $U_1$, \dots, $U_n$ be finitely many open subsets of the topology. Then
  their intersection $U_1 \cap \dotsb \cap U_n$
  is again an open subset of the topology.
  
  A \emph{topological space $X$} is a set $X$ together with a topology $\mathfrak U^0(X)$
  on $X$. An \emph{open set of a topological space $X$} is an open set of the topology
  of $X$.
  A \emph{point $p \in X$ of a topological space $X$} is an element of the
  underlying set of $X$.
\end{dfn}

As the union of an empty family of subsets of a set $X$ is the empty subset,
the empty subset is open with respect to any topology on $X$.
As the intersection of an empty family of subsets of $X$ is the whole set $X$,
the whole set $X$ is open with respect to any topology on $X$.

Let $X$ be a topological space.
A \emph{neighborhood of a point $p$ in $X$} is a subset $G$
of the underlying set of $X$ such that there exists an open subset $U$ of $X$ with
$p \in U \subseteq G$. The set of neighborhoods of $p$ is denoted by
$\mathfrak U(p, X)$. A neighborhood that is at the same time an open subset
is an \emph{open neighborhood}. The set of open neighborhoods of $p$ is
denoted by $\mathfrak U^0(p, X)$. A subset $U$ of the underlying set of the
topological space $X$ is open if and only if it is a neighborhood of every point
$p \in U$. A \emph{neighborhood of a subset $A$ of $X$} is a subset $G$ that is
a neigborhood of every point $p \in A$. The set of neighborhoods of $A$ is denoted
by $\mathfrak U(A, X)$.

An \emph{closed subset $Z \subseteq X$ of $X$} is a subset of the underlying
set of $X$ such that its complement $X \setminus Z$ is open. Let $Y$ be an
arbitrary subset of $X$. The \emph{closure $\overline Y$ of $Y$ in $X$} is the
smallest closed subset of $X$ containing $Y$, that is
\begin{equation}
  \overline Y = \bigcap_{\text{$Y \subseteq Z \subseteq X$ is a closed subset}} Z
\end{equation}

Let $Y$ be any subset of the underlying set of $X$. Set
\[
  \mathfrak U^0(Y) \coloneqq \Set{Y \cap U: U \in \mathfrak U^0(X)}
\]
This is a topology, the \emph{subspace topology}. Endowed with this topology,
$Y$ canonically becomes a topological space itself. Topological spaces of this form
are called \emph{subspaces of $X$}. If $Y$ is a closed subset of $X$, a subset of
$Y$ is closed in $Y$ if and only if it is closed in $X$. In particular, the closure
of a subset of $Y$ in $Y$ is the same as its closure in $X$.

The cartesian space $\set R^n$ is canonically a topological space with the
topology defined by
\begin{equation}
  \label{eq:cartesian_topology}
  \mathfrak U^0(\set R^n) = \Set{U \subseteq \set R^n : \forall p \in U \,
  \exists \epsilon \in \set R^+: U_\epsilon(p) \subseteq U},
\end{equation}
where
\[
  U_\epsilon(p) \coloneqq \Set{q \in \set R^n : \norm{q - p} < \epsilon}
\]
for an $\epsilon > 0$ is the standard \emph{$\epsilon$-neighborhood}. Its 
closure under the so-defined topology is given by
\[
  \overline U_{\epsilon(p)} = B_\epsilon(p) \coloneqq
  \Set{q \in \set R^n : \norm {q - p} \leq \epsilon}.
\]

Given a pair of topological space $X$ and $Y$, the product
$X \times Y = \Set{(x, y) : x \in X, y \in Y}$ carries a canonical topology
defined by
\begin{equation}
  \label{eq:product_topology}
  \mathfrak U^0(X \times Y) = \Set{W \subseteq X \times Y \forall (x, y) \in W \,
  \exists U \in \mathfrak U^0(x, X), V \in \mathfrak U^0(y, Y):
  U \times V \subseteq W}.
\end{equation}
Endowed with this topology, $X \times Y$ is called the \emph{product space} or
\emph{topological product of $X$ and $Y$}. If $A$ is a subspace of $X$ and
$B$ is a subspace of $Y$, the subspace $A \times B$ of $X \times Y$ carries the
product topology of the topological product between $A$ and $B$.

\section{Continuous maps}
\label{sec:continuity}

\begin{dfn}
  A \emph{continuous map $f\colon X \to Y$ from an topological space $X$ to
  a topological space $Y$} is a mapping $f\colon X \to Y$ such that
  \[
    \forall V \in \mathfrak U^0(Y) : f^{-1}(V) \in \mathfrak U^0(X).
  \]
\end{dfn}

Let $X$ be a topological space and let $\phi\colon X \to \set R$ be a continuous map.
The \emph{support $\supp \phi$ of $\phi$} is the closed subset
\[
  \Set{p \in X : \forall G \in \mathfrak U(p, X): \phi|G \not\equiv 0}.
\]
It is given by
\begin{equation}
  \label{eq:support}
  \supp \phi = \overline{\Set{p \in X : \phi(p) \neq 0}}.
\end{equation}

If $X$ and $Y$ are two topological spaces, the two projection maps
$\pr_1\colon X \times Y \to X, (x, y) \mapsto x$ and
$\pr_2\colon X \times Y \to Y, (x, y) \mapsto y$ are continuous.

\section{Hausdorff spaces}
\label{sec:hausdorff_spaces}

\begin{dfn}
  A \emph{Hausdorff space $X$} is a topological space $X$ such that for every
  pair $p$ and $q$ of points of $X$ with $p \neq q$, there are neighborhoods $G$
  of $p$ and $H$ of $q$, respectively, such that $G \cap H = \emptyset$.
\end{dfn}
A subspace of a Hausdorff space is again a Hausdorff space. Any finite subset of
a Hausdorff space is a closed subset. The topological product of two Hausdorff
spaces is again a Hausdorff space.

Cartesian space $\set R^n$ is an example for a Hausdorff space.

The importance of Hausdorff spaces partially stems from the fact that
solutions sets of equations with values in Hausdorff spaces are closed: Let
$f$ and $g\colon X \to Y$ be two continuous maps from an arbitrary topological
space into a Hausdorff space. Then
\[
  \Set{x \in X: f(x) = g(x)}
\]
is a closed subset of $X$.

\section{Connected spaces}

\begin{dfn}
  A topological space $X$ is \emph{connected} if
  \[
    X = \coprod_{i \in I} X_i \implies \exists i \in I: X = X_i
  \]
  whenever $(X_i)_{i \in I}$ is a family of open subsets of $X$.
\end{dfn}

Note that the empty set $\emptyset$ viewed as a topological space is never
connected: it is the empty disjoint union of no open subspace, so there exists
no $i \in I$ with $\emptyset = X_i$.

To show that a subset $Y$ of a connected space $X$ is all of $X$, it is enough
to show that $Y$ is non-empty and both an open and closed subset of $X$.

A subspace of $\set R$ is connected if and only it is a non-empty interval.

\section{Normal spaces}

\begin{dfn}
  A Hausdorff space $X$ is \emph{normal} such that for any disjoint closed
  subsets $A$ and $B$ of $X$ there exists neighborhoods $U \in \mathfrak U(A, X)$
  and $V \in \mathfrak U(B, X)$ with $U \cap V = \emptyset$.
\end{dfn}

An open cover $(U_i)_{i \in I}$ of a topological space $X$ is \emph{locally finite}
if every point of $X$ possesses a neighborhood $G$ such that there only finitely
many $i \in I$ with $G \cap U_i \neq \emptyset$.
\begin{thm}[Shrinking lemma]
  \label{prop:shrinking_lemma}
  For any locally finite open cover $(U_i)_{i \in I}$ of a normal Hausdorff space $X$, there
  exists a locally finite open cover $(V_i)_{i \in I}$ with
  \[
    \forall i \in I : V_i \subseteq \overline{V_i} \subseteq U_i.
  \]
\end{thm}

\section{Compact spaces}
\label{sec:compact}

A family $(U_i)_{i \in I}$ of open subsets of a topological space $X$ is an
\emph{open cover of $X$} if
\[
  X = \bigcup_{i \in I} U_i.
\]
A \emph{subcover} of an open cover $(U_i)_{i \in I}$ of $X$
is an open cover of the form $(U_i)_{i \in I'}$ for $I' \subseteq I$. An open
cover $(U_i)_{í \in I}$ of $X$ is \emph{finite} if the index set $I$ is finite.

If $Y$ is a subspace of a topological space $X$ and $(U_i)_{i \in I}$ is a family
of open subsets of $X$ such that $(Y \cap U_i)_{i \in I}$ is an open cover of
$Y$, we often say for simplicity that $(U_i)_{i \in I}$ is an open cover of $Y$.

\begin{dfn}
  A \emph{compact space} is a Hausdorff space $X$ such that every open cover of
  $X$ has a finite subcover.
\end{dfn}

A subset $K$ of the underlying set of a Hausdorff space $X$ is \emph{compact}
if it is a compact space when endowed with the subspace topology. It is \emph{relatively
compact in $X$} if its closure in $X$ is compact.

A compact subspace of a Hausdorff space $X$ is always a closed subset of $X$. A closed
subset of a compact space is again compact. The product of two compact spaces
is again a compact space.

\begin{thm}
  A subset $K$ of cartesian space $\set R^n$ is compact if and only if it is closed
  and there exists an $R \in \set R_+$ such that $K \subseteq \Set{p \in \set R^n : \norm p \leq R}$.
\end{thm}

The latter condition on $K$ says that $K$ is \emph{bounded}.

\section{Locally compact spaces}
\label{sec:locally_compact}

\begin{dfn}
  A Hausdorff space $X$ is \emph{locally compact} if each $p \in X$ has a
  neighborhood $U \in \mathfrak U(p, X)$, which is compact.
\end{dfn}
The condition of being locally compact for a Hausdorff space $X$ is equivalent to
requiring that every point $p \in X$ possesses an open neighborhood $U \in \mathfrak U(p, X)$,
which is relatively compact in $X$.

\begin{prop}
  \label{prop:locally_compact}
  On a locally compact Hausdorff space $X$, every neighborhood $G$ of a point $p \in X$
  contains a compact neighborhood $K \in \mathfrak U(p, X)$.
\end{prop}

The product of two locally compact Hausdorff spaces is again a locally compact
Hausdorff space.

\section{Paracompact Hausdorff spaces}
\label{sec:paracompactness}

A \emph{refinement} of an open cover $(U_i)_{i \in I}$ of a topological space $X$
is an open cover $(V_j)_{j \in J}$ of $X$ such that
\[
  \forall j \in J \, \exists i \in I: V_j \subseteq U_i.
\]

\begin{dfn}
  A \emph{paracompact Hausdorff space} is a Hausdorff space $X$ such that every
  open cover of $X$ has a locally finite refinement.
\end{dfn}

\begin{prop}
  \label{prop:paracompact_spaces}
  A paracompact Hausdorff space is normal.
\end{prop}

\begin{thm}
  The topological product of two locally compact paracompact Hausdorff spaces
  is again a (locally compact) paracompact Hausdorff space. 
\end{thm}

Cartesian space $\set R^n$ is an example of a paracompact Hausdorff space.

\section{Problems}

\begin{xca}
  Let $X$ be a topological space. Prove that a subset $U$ of the underlying set
  of $X$ is open if and only if it is a neighborhood of every point $p \in U$.
\end{xca}

\begin{xca}
  Prove that the subspace topology of a subset of the underlying set of a topological
  space is in fact a topology.
\end{xca}

\begin{xca}
  Let $Y$ be a closed subset of a topological space $X$. Let $Z$ be a subset of
  $Y$. Show that $Z$ is closed in $Y$ if and only if it is closed in $X$.
\end{xca}

\begin{xca}
  Show that \prettyref{eq:cartesian_topology} defines a topology on $\set R^n$.
\end{xca}

\begin{xca}
  Show that \prettyref{eq:product_topology} defines a topology on the cartesian
  product of the topological spaces $X$ and $Y$.
\end{xca}

\begin{xca}
  Show that the closure of a standard $\epsilon$-neighborhood in $\set R^n$
  for the canonical topology is given by
  \begin{equation}
    \overline U_{\epsilon(p)} = \Set{q \in \set R^n : \norm {q - p} \leq \epsilon}.
  \end{equation}
\end{xca}

\begin{xca}
  Let $A$ and $B$ be subspaces of the topological spaces $X$ and $Y$, respectively.
  Show that the subspace $A \times B$ of $X \times Y$ carries the
  product topology of the topological product between $A$ and $B$.
\end{xca}

\begin{xca}
  Let $X$ be a topological space. Prove that the support of a continuous map
  $\phi\colon X \to \set R$ is indeed closed and that the formula \prettyref{eq:support}
  holds.
\end{xca}

\begin{xca}
  Let $X$ and $Y$ be two topological spaces. Show that the two projection maps
  $\pr_1$ and $\pr_2$ defined on the topological product $X \times Y$ are
  continuous.
\end{xca}

\begin{xca}
  Let $X$ be a connected topological space and $\emptyset \neq Y \subseteq X$
  a non-empty subset. Assume that $Y$ is both open and closed in $X$. Show that
  $Y = X$.
\end{xca}

\begin{xca}
  Prove that the connected subspaces of $\set R$ are exactly the non-empty
  intervals.
\end{xca}

\begin{xca}
  Prove that a subspace of a Hausdorff space is again a Hausdorff space.
\end{xca}

\begin{xca}
  Prove that a finite subset of a Hausdorff space is a closed subset.
\end{xca}

\begin{xca}
  Prove that a product of two Hausdorff spaces is again a Hausdorff space.
\end{xca}

\begin{xca}
  Let $f$ and $g\colon X \to Y$ be two continuous maps from an arbitrary  
  topological space into a Hausdorff space. Prove that
  \[
    \Set{x \in X: f(x) = g(x)}
  \]
  is a closed subset of $X$.
\end{xca}

\begin{xca}
  Prove that a compact subspace $K$ of a Hausdorff space $X$ is always a closed subset
  of $X$.
\end{xca}

\begin{xca}
  Prove that a closed subset $Z$ of a compact space $K$ is again compact.
\end{xca}

\begin{xca}
  Prove that the product of two compact spaces is again a compact space.
\end{xca}

\begin{xca}
  Prove that a Hausdorff space $X$ is locally compact if and only if every point
  $p \in X$ possesses an open neighborhood $U \in \mathfrak U(p, X)$ such that
  $U$ is relatively compact in $X$:
\end{xca}

\begin{xca}
  Prove that the product of two locally compact Hausdorff spaces is again a
  locally compact Hausdorff space.
\end{xca}

\begin{xca}
  Show that the canonical topology on $\set R^n$ makes cartesian space into
  a paracompact Hausdorff space.
\end{xca}


  % Topological spaces
  % Continuous maps
  % Hausdorff spaces
  % Normal spaces
  % Compact spaces
  % Locally compact spaces
  % Paracompact Hausdorff spaces
  
  \chapter{Analysis}

\section{Bump functions}
\label{sec:bump_functions}

\begin{thm}
  \label{thm:bump_function}
  For any two real numbers $0 < r < R$ there exists a smooth
  function $\phi\colon \set R^n \to \set R$ with
  \begin{align*}
    \forall p \in \set R^n: \phi(p) & \ge 0, &
    \forall p \in \set R^n: \norm p \leq r \implies \phi(p) & = 1, &
    \forall p \in \set R^n: \norm p \geq R \implies \phi(p) & = 0.
  \end{align*}
\end{thm}

A function as in the theorem is called a \emph{bump function}.

\section{Hadamard's lemma}

A subset $U$ of $n$-dimensional Euclidean space $\set R^n$ is \emph{star-shaped with
respect to a point $p \in U$} if
\[
  \forall x \in U \forall 0 \leq t \leq 1 :
  (1 - t) \, a + t \, x \in U.
\]
Any neighborhood of a point $p \in \set R^n$ contains an open neighborhood
star-shaped with respect to the point $p$ as the standard $\epsilon$-neighborhoods
are star-shaped with respect to their center (in fact, with respect to any point
of their interior).

\begin{thm}
  \label{thm:hadamard}
  Let $\phi$ be a smooth function defined on an open subset $U$ of $n$-dimensional
  Euclidean space $\set R^n$ that is star-shaped with respect to a point $a \in U$.
  Then there exist smooth functions $g_1$, \dots, $g_n$ on $U$ such that
  \[
    \forall x \in U: \phi(x) = \phi(a) + \sum_{i = 1}^n (x_i - a_i) \, g_i(x).
  \]
\end{thm}

  % Bump functions
	
\end{document}
