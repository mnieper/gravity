\chapter{Spacetime}

\section{Introduction}

The fundamental notion of special relativity is that of an \emph{event}.
In pre-relativity, things located at a specific point in space and time,
for example the start of the Saturn V rocket launching the Apollo 11
spaceflight, are described by a unique time in Newton's absolute time and
a unique place in Euclidean absolute space. The notion of an event combines
these two qualities into one, which is essential for special relativity
as space and time by themselves lose their absoluteness.

\emph{Spacetime} is the set of all possible events. Any event that can
be imagined to happen is an element or a
\emph{point} in spacetime. Mathematically, spacetime has more
structure than simply being a set: A point in this set, that is an event, is
usually described by four scalars, one time and three space coordinates,
making spacetime four-dimensional. For a general set, however, there is no
well-defined notion of coordinates or dimension. The theory of manifolds, which
will be presented in the next section, is the correct mathematical setting in
which notions like coordinates and dimension make sense.

\section{Manifolds}
\label{sec:manifolds}

An \emph{$n$-dimensional chart $x = (x, U)$} on a set $M$, whose elements we call
\emph{points}, is an injective map
$x = (x_1, \dotsc, x_n)\colon U \to \set R^n$ onto an open subset of $\set R^n$ defined on a subset
$U$ of $M$, the \emph{domain of definition of $x$}. Given a point $p$ lying in the
domain of definition $U$, the $n$ scalars $x_1(p)$, \dots, $x_n(p)$ are the
\emph{coordinates of $p$ with respect to the chart $x$}. A
family $\Set{(x_i, U_i)}_{i \in I}$ of charts on $M$ \emph{covers} $M$ if $M \subseteq
\bigcup_{i \in I} U_i$ that is if every point $p$ of $M$ lies in at least domain
of definition of the charts $x_i$. In order to be able to describe events in
spacetime by four coordinates, we postulate that \emph{there is a distinguished
family of four-dimensional charts that cover spacetime}.

Given two $n$-dimensional charts $(x, U)$ and $(y, V)$ of a set $M$, the map
\[
  y \circ x^{-1}|x(U \cap V)\colon x(U \cap V) \to y(U \cap V)
\]
is called the
\emph{coordinate transformation from $x$ to $y$}. If this map is a diffeomorphism
between open subsets of $\set R^n$, the charts $x$ and $y$ are said to be
\emph{compatible}.
An \emph{$n$-dimensional atlas of $M$} is a family of pairwise compatible
$n$-dimensional charts of $M$ that covers $M$. In order to be able to employ 
analytic methods, we extend our postulate above by postulating that
\emph{there is a distinguished four-dimensional atlas of spacetime}.

Let $\mathfrak A$ be any $n$-dimensional atlas of a set $M$, and let $x$ and $y$
be two arbitrary $n$-dimensional charts of $M$. If $x$ and $y$ are each compatible
with each chart in $\mathfrak A$, they are compatible with each other. Therefore,
every atlas $\mathfrak A$ can be uniquely enlarged to a \emph{maximal atlas} in
which it is contained by adding all $n$-dimensional charts to $\mathfrak A$ that
are compatible with each in chart in $\mathfrak A$. We also say that the maximal
atlas is \emph{generated} by the charts contained in $\mathfrak A$.

An \emph{$n$-dimensional premanifold $M$} is set $M$ together
with a maximal $n$-dimensional atlas $\mathfrak U^\infty(M)$ of $M$. The set of
all charts $(x, U)$ in $\mathfrak U^\infty(M)$ that contains a given point $p \in M$ is 
denoted by $\mathfrak U^\infty(p, M)$. A \emph{chart of $M$} is a chart in the
maximal atlas $\mathfrak U^\infty(M)$.
An \emph{atlas of the premanifold $M$} is any atlas of $M$ which is contained in
the maximal atlas $\mathfrak U^\infty(M)$. With these terms, we can say that
\emph{spacetime is a four-dimensional premanifold}.

Using the atlas of a premanifold $M$, one can define the notion of neighborhoods
of points on $M$: A subset $G$ of the underlying set of $M$ is an \emph{open subset
of $M$} if $x(U \cap G)$ is an open subset for each chart $(x, U)$ of $M$. The
system of these open subsets of $M$ is a topology (see~\prettyref{sec:topological_spaces})
on the underlying set of $M$, the \emph{canonical topology of $M$}, so that $M$
is a topological space in a canonical way.

For any chart $(x, U)$ of the $n$-dimensional premanifold, a subset $V$ of $U$
is open in $M$ if and only if $x(V)$ is open in $\set R^n$. In particular, $U$ is open.
The map $x\colon U \to \set R^n$ is continuous in the sense of maps between
topological spaces.

Generally, the so-defined topology is ill-behaved, however;
points may not be distinguishable by the topology and one may need infinitely
many charts to cover even small neighborhoods of points. Excluding these cases
by adding technical conditions on the underlying topological space leads to the
final definition:
\begin{dfn}
  An \emph{$n$-dimensional manifold $M$} is an $n$-dimensional premanifold $M$
  whose underlying topological space is a paracompact Hausdorff space
  (see~\prettyref{sec:paracompactness}).
\end{dfn}
We extend our postulate from above by postulating that \emph{spacetime is a
four-dimensional manifold}.

The most basic example of an $n$-dimensional manifold is the $n$-dimensional 
cartesian space $\set R^n$. Its canonical atlas is the maximal atlas which is generated
by the identity map $\id_{\set R^n}\colon \set R^n \to \set R^n$ viewed
as an $n$-dimensional chart. The topology defined by this atlas is, of course,
the canonical topology of $\set R^n$, which is the topology of a paracompact
Hausdorff space.

Requiring that the underlying topological space of spacetime is a paracompact
Hausdorff space has a number of pleasant consequences: For example, the underlying
topological spaces of manifolds are normal, which follows from \prettyref{prop:paracompact_spaces}.
Furthermore, we have:
\begin{prop}
  The underlying space of an $n$-dimensional manifold $M$ is locally compact.
\end{prop}
(For the definition of local compactness, see \prettyref{sec:locally_compact}.)

\begin{proof}
  For $p \in M$ choose a chart $(x, U) \in \mathfrak U^\infty(p, M)$. By openness
  of $x(U)$ in $\set R^n$, there exists an $\epsilon > 0$ such that $U_\epsilon(x(p))
  \subseteq x(U)$. We claim that
  \[
    K \coloneqq x^{-1}(\overline{U_{\frac \epsilon 2}(x(p))})
  \]
  is a compact neighborhood of $p$:
  
  The subset $K$ is a neighborhood of $p$ as
  $p \in x^{-1}(U_{\frac \epsilon 2}(x(p))) \subseteq K$ and
  because $x^{-1}(U_{\frac 1 2 \epsilon}(x(p)))$ is an open subset of $M$.

  To show that $K$ is compact, let $(U_i)_{i \in I}$ be a family of open subsets
  of $U$ that cover $K$. By definition of the topology of $M$, the images $(x(U_i))_{i \in I}$
  form an open cover of the compact space $\overline{U_{\frac \epsilon 2}(x(p))}$.
  Thus there is a finite subset $J \subset I$ such that $(x(U_i))_{i \in J}$ is
  an open cover of $\overline{U_{\frac \epsilon 2}(x(p)}$. It follows that
  $(U_i)_{i \in J}$ is a finite open cover of $K$.
\end{proof}

Often, we have to restrict our attention to small pieces (that is, open subsets)
of a given $n$-dimensional manifold $M$, for example spacetime. This can be
done as follows: Let $G$ be an open subspace of $M$. Let $\mathfrak U^\infty(G)$
be the unique maximal $n$-dimensional atlas of $G$ that contains the atlas
\[
  \Set{(x|U \cap G, U \cap G) : (x, U) \in \mathfrak U^\infty(M)}.
\]
Then $G$ becomes an $n$-dimensional manifold itself, whose underlying topological
space is the subspace $G$ of $M$. An $n$-dimensional manifold of this form is
called an \emph{open submanifold of $M$}. The domains of definitions of the charts of
$M$ are open submanifolds of $M$.

\section{Functions on a manifold}
\label{sec:functions}

As we have postulated, we can ``measure'' each event in spacetime by giving the
scalar values of four coordinates (after choosing a chart). Each coordinate can
be thought of as a scalar field that assigns to each point (in the domain of
definition of its chart) a scalar value. This notion is generalised as follows:

\begin{dfn}
  A \emph{(smooth) function $\phi\colon M \to \set R$} on an $n$-dimensional
  manifold $M$ is a mapping $\phi$ from the underlying set of $M$ to the reals
  such that for each chart $(x, U)$ of $M$ the map
  \[
    \phi \circ x^{-1}\colon x(U) \to \set R
  \]
  is smooth.
\end{dfn}

In order to show that a map $\phi\colon M \to \set R$ is a function in the above
sense, it suffices that
for each point $p \in M$ there exists a chart $(x, U) \in \mathfrak U^\infty(p, M)$
such that $\phi \circ x^{-1}\colon x(U) \to \set R$ is smooth.

Every function is continuous with respect to the underlying
topologies of $M$ and $\set R$, respectively. Let $\phi$ be a function on $M$
and $G$ an open submanifold. The \emph{restriction $\phi|G$ of $\phi$ to $G$},
given by
\[
  \phi|G\colon G \to M, p \mapsto f(p)
\]
is a function on the manifold $G$. For any chart $(x, U)$ of $M$, the coordinate
functions $x_1$, \dots, $x_n\colon U \to M$ are smooth functions on the open
submanifold $U$.

All functions on $M$ form an algebra (over the reals), denoted by
$\mathcal C^\infty(M)$, where addition and multiplication are defined point-wise.
The function with constant value $c \in \set R$ is often denoted by $\underline c$.
The sets of functions of the open submanifolds of $M$ fulfill the \emph{sheaf
condition}, that is for every open cover $(U_i)_{i \in I}$ of $M$ and functions
$\phi_i \in \mathcal C^\infty(U_i), i \in I$ one has
\[
  \left(\forall i, j \in I : \phi_i|U_i \cap U_j = \phi_j|U_i \cap U_j\right)
  \implies \exists! \phi \in \mathcal C^\infty(M) \, \forall i \in I :
  \phi|U_i = \phi_i.
\]
In other words, we can uniquely glue functions along open submanifolds.

If $\Phi\colon \set R \to \set R$ is any smooth function between the reals, 
the composition $\Phi \circ \phi\colon M \to \set R$ is a function on $M$
whenever $\phi\colon M \to \set R$ is a function.

A manifold possesses many functions in a sense made precise by the following theorem,
which relies essentially on the paracompactness of the manifold (for the notion
of the support $\supp$ of a function, see \prettyref{sec:continuity}):
\begin{thm}
  \label{thm:paracompact_manifold}
  Every open cover $(U_i)_{i \in I}$ of a manifold $M$ has a
  \emph{subordinate
  partition $(\lambda_i)_{i \in I}$ of unity}, which is a family of functions
  $\lambda_i\colon M \to \set R$ with the following properties:

  \paragraph{Range}
  For all $i \in I$ and $p \in M$, one has $0 \leq \lambda_i(p)
  \leq 1$.
  
  \paragraph{Support}
  For all $i \in I$, one has $\supp \lambda_i \subseteq U_i$.
  
  \paragraph{Local finiteness}
  Every $p \in M$ posesses a neighborhood $G \in \mathfrak U^0(p, M)$ such that there
  are only finitely many $i \in I$ with $\supp \lambda_i \cap G \neq \emptyset$.
  
  \paragraph{Normalization}
  For every $p \in M$, the equality $\sum_{i \in I} \lambda_i(p) \equiv 1$ holds.
\end{thm}

The proof relies on the following lemma, which is also of independent interest:
\begin{lem}
  \label{lem:paracompact_manifold}
  Let $K$ be a compact subspace of an $n$-dimensional manifold $M$. For any open neighborhood
  $G$ of $K$ in $M$, there exists a function $\phi\colon M \to \set R$ with
  \begin{align}
    \label{eq:paracompact_manifold}
    \forall p \in M: \phi(p) & \ge 0, &
    \forall p \in K: \phi(p) & > 0, &
    \supp \phi & \subseteq G.
  \end{align}
\end{lem}

\begin{proof}[Proof of \prettyref{lem:paracompact_manifold}]
  For any $p \in K$ choose a chart $(x, U) \in \mathfrak U^\infty(p, M)$. 
  By the openness of $x(U)$ in $\set R^n$, there exists an $\epsilon > 0$ with
  $U_\epsilon(x(p)) \subseteq x(U \cap G)$. Choose a bump function $\psi\colon \set R^n \to
  \set R$ (see \prettyref{sec:bump_functions}) such that $\psi(u) \ge 0$ for all
  $u \in \set R^n$, $\psi(u) = 1$ for
  $\norm u \leq \frac \epsilon 3$ and $\psi(u) = 0$ for $u \ge \frac {2 \epsilon} 3$.
  By the sheaf condition, 
  \[
    \phi_p\colon M \to \set R, q \mapsto \begin{cases}
      \psi(x(q) - x(p)) & \text{if $q \in U$}, \\
      0 & \text{if $q \in M \setminus x^{-1}(\overline{U_{\frac {2 \epsilon} 3}(x(p))})$}
    \end{cases}
  \]
  defines a function on $M$ as $U$ and $M \setminus x^{-1}(\overline{U_{\frac {2 \epsilon} 3}(x(p))})$
  form an open cover of $M$. This function has the properties
  \begin{align*}
    \forall q \in M: \phi_p(q) & \ge 0, &
    \phi_p(p) & > 0, &
    \supp \phi_p & \subseteq G.
  \end{align*}
  
  The open subsets $U_p \coloneqq \Set{q \in M : \phi_p(q) > 0}$ with $p \in K$ cover $K$. By
  compactness of $K$, there exists a finite subset $A \subseteq K$ such that
  $K$ is covered by $(U_p)_{p \in K}$. By construction, the function
  $\phi = \sum_{p \in A} \phi_p$
  fulfills \prettyref{eq:paracompact_manifold}.
\end{proof}

\begin{proof}[Proof of \prettyref{thm:paracompact_manifold}]
  For every $p \in M$ choose by local compactness of $M$ a relatively compact neighborhood $G_p
  \in \mathfrak U(p, M)$. By the covering property, there exists an $i \in I$ with
  $p \in U_i$. The intersection $G_p \cap U_i$ is again relatively compact in $M$,
  so we may assume that already $G_p \subseteq U_i$. So  
  $(G_p)_{p \in M}$ is a refinement of the cover $(U_i)_{i \in I}$.
  Let $(V_j)_{j \in J}$ be a locally finite refinement of $(G_p)_{p \in M}$. In
  particular, $(V_j)_{j \in J}$ is a locally finite refinement of $(U_i)_{i \in I}$
  and each $V_j$ is relatively compact in $M$. By the shrinking lemma, \prettyref{prop:shrinking_lemma},
  and the normality of the underlying topological space of $M$, there exists
  another open cover $(V'_j)_{j \in J}$ of $M$ with
  $\overline{V'_j} \subseteq V_j$ for all $j \in J$.
  
  For every $j \in J$, choose by compactness of $\overline{V'_j}$ and
  \prettyref{lem:paracompact_manifold} a function
  $\phi_j \in \mathcal C^\infty(M)$ with
  \begin{align*}
    \forall p \in M: \phi_j(p) & \ge 0, &
    \forall p \in V'_j: \phi_j(p) & > 0, &
    \supp \phi_j \subseteq V_j.
  \end{align*}
  By the local finiteness of the open cover $(V_j)_{j \in J}$, the sum
  $\phi \coloneqq \sum_{j \in J} \phi_j$ is locally a finite sum and thus defines
  a function on $M$ by the sheaf condition. By the covering property of
  $(V'_j)_{j \in J}$,
  one has $f(p) > 0$ for all $p \in M$.
  
  As $(V_j)_{j \in J}$ is a refinement of the cover $(U_i)_{i \in I}$, there exists a
  map $\alpha\colon J \to I$ with $V_j \subseteq U_{\alpha(i)}$ for all $j \in J$.
  For each $i \in I$ set $J_i \coloneqq \alpha^{-1}(i)$, so $(J_i)_{i \in I}$ becomes
  a partition of $I$. For all $i \in I$, set
  $U'_i \coloneqq \bigcup_{j \in J_i} V_j \subseteq U_i$ and finally
  \[
    \lambda_i = \sum_{j \in J_i} \frac{\phi_i}{\phi}.
  \]
  By a similar argument as above, the sum on the right hand side is locally finite
  and, thus, $\lambda_i$ is a function on $M$ with $\lambda_i(p) \ge 0$ for all
  $p \in M$.
  
  By construction, $\supp \lambda_i \subseteq \bigcup_{j \in J_i} V_j = U'_j \subseteq
  U_i$, which proves the support axiom of a partition of unity. The covering
  $(U'_i)_{i \in I}$ is locally finite as the covering $(V_j)_{j \in J}$ is locally
  finite; this uses the disjointness of the $J_i$. From $\supp \lambda_i \subseteq U'_i$
  for all $i \in I$, the local finiteness axiom follows. By construction,
  $\sum_{i \in I} \lambda_i \equiv 1$, which is the normalization axiom of a
  partition of unity.
  From this, the range axiom follows as we already know that $\lambda_i(p) \ge 0$
  for all $p \in M$.
\end{proof}

The existence of partitions of unity on manifolds implies that functions can
be extended in the following sense:

\begin{cor}
  Let $\phi$ be a function defined on an open
  neighborhood $G$ of a point $p$
  in a manifold $M$. Then there exists a function $\widehat \phi \in \mathcal C^\infty(M)$
  with $\supp \widehat\phi \subseteq G$ and
  such that $\widehat \phi$ coincides with $\phi$ on a neighboorhood $U \subseteq G$
  of $p$ in $M$.
\end{cor}
The function $\widehat \phi$ is called an \emph{extension of $\phi$
by zero away from $p$}.

\begin{proof}
  By local compactness (see \prettyref{prop:locally_compact}),
  there exists a compact neighborhood $K \in \mathfrak U(p, G)$.
  Choose a partition $(\lambda, \mu)$ of unity subordinate to the open cover
  $(G, M \setminus K)$ of $M$. By the sheaf condition,
  \[
    \hat\phi\colon M \to \set R,
    \begin{cases}
      \lambda(p) \cdot \phi(p) & \text{if $p \in G$} \\
      0 & \text{if $p \in M \setminus \supp \lambda$}
    \end{cases}
  \]
  is a well-defined function on $M$ with $\supp \hat\phi \subseteq G$ and
  which coincides with $\phi$ on $K$.
\end{proof}

For any two points $p$, $q \in M$ with $p \neq q$, there exists an open
neighborhoods $G \in \mathfrak U^0(p, M)$ with $q \notin G$. Extending the
constant function $\underline 1|G$ by zero away from $p$ yields a function,
which is $1$ on $p$ and $0$ on $q$. Thus, we have
\begin{equation}
  \label{eq:point_separation}
  \forall p, q \in M : p \neq q \implies
  \exists \lambda \in \mathcal C^\infty(M) : \lambda(p) = 1, \lambda(q) = 0.
\end{equation}

If we denote by $p^*$ for all $p \in M$ the algebra homomorphism
\begin{equation}
  p^*\colon \mathcal C^\infty(M) \to \set R, \phi \mapsto \phi(p),
\end{equation}
we can reformulate \prettyref{eq:point_separation} by saying that the map $p \mapsto p^*$ is injective,
that is the algebra of functions \emph{separate points}. The algebra of functions
on spacetime is therefore a full set of observables: for any two distinct events
there is a function that takes different values on both events.

\section{Morphisms}
\label{sec:morphisms}

A physical body traces out a curve of events in spacetime $M$, namely those events
where an observer meets the physical body, its \emph{world line}. If the physical
body carries a clock with it, each point of its world line is parametrized by
a scalar, the clock's time measured at that event. In other words, the path of
the physical body in spacetime together with its clock defines a map $J \to M$, 
where $J$ is an (open) interval. Both the domain and the target of this map is a 
manifold, where $J$ is viewed as an open submanifold of the reals. A physical
body does not jump through spacetime, so the map will be continuous.
In order to employ analytical methods, it is sensible to assume moreover that
this map is a morphism according to the following definition:

\begin{dfn}
  A \emph{morphism $f\colon M \to N$} between two manifolds $M$ and $N$ is a
  continuous map $f\colon M \to N$ such that for each pair of charts
  $(x, U) \in \mathfrak U^\infty(M)$ and $(y, V) \in \mathfrak U^\infty(N)$ the
  composition
  \[
    y \circ f \circ x^{-1}|x(f^{-1}(V) \cap U)\colon x(f^{-1}(V) \cap U) \to y(V) 
  \]
  is a smooth map (between open subsets of cartesian spaces).
\end{dfn}

In accordance with our above wording, a \emph{curve $\alpha\colon J \to M$} is a
morphism where $J$ is an open interval viewed as a submanifold of $\set R$. Thus
a world line of a physical body becomes a curve in this sense
by endowing it with a clock.

A map $f\colon M \to N$ between manifolds is a morphism if and only if
for all functions $\psi \in \mathcal C^\infty(N)$ the \emph{pullback of $\psi$
by $f$}, given by
\begin{equation}
  f^{-1}\psi\colon M \to \set R, p \mapsto \psi(f(p)),
\end{equation}
is a function on $M$. (The pullback itself is a map
\begin{equation}
  f^{-1}\colon \mathcal C^\infty(N) \to \mathcal C^\infty(M)
\end{equation}
of algebras.)

Thus, a map $\phi\colon M \to \set R$ defined on a manifold $M$ is a
morphism if and only if it is a function. Further, a map $f\colon G \to H$ between
open subsets of cartesian spaces is a morphism if and only if it is a smooth
map in the sense of calculus.

The identity $\id_M\colon M \to M$ of $M$ is a morphism. The composition $g \circ f$
of two morphisms $f\colon M \to N$ and $g\colon N \to P$ between manifolds is
again a morphism. The manifolds together with the morphisms between them thus
form a \emph{category}. A morphism $f\colon M \to N$ between manifolds that
is bijective and whose inverse $f^{-1}\colon N \to M$ is again a morphism is
called a \emph{diffeomorphism between $M$ and $N$}. In a categorical sense,
diffeomorphisms are exactly the isomorphisms.

The \emph{inclusion $i\colon U \to M, p \mapsto p$} of an open submanifold $U$
of $M$ is a morphism. Thus, the restriction $f|U = f \circ i$ of a morphism
$f\colon M \to N$ to $U$ is again a morphism.

\section{Product manifolds}

When investigating possible universe, that is possible spacetimes, we will have
to construct manifolds. One important construction, which constructs a manifold
out of simpler ones is the product. Let $M$ and $N$ be two manifolds of dimensions
$m$ and $n$, respectively. Given any chart $(x, U) \in \mathfrak U^\infty(M)$
and any chart $(y, V) \in \mathfrak U^\infty(N)$, the map
\[
  x \times y\colon U \times V \to \set R^{m + n}, (p, q) \mapsto (x(p), y(q))
\]
defines an $(m + n)$-dimensional chart on the cartesian product
\[
  M \times N \coloneqq \Set{(p, q) : p \in M, q \in N}.
\]

Let $\mathfrak U^\infty(M \times N)$ be the unique maximal atlas of $M \times N$
that contains the atlas
\[
  \Set{(x \times y, U \times V): (x, U) \in \mathfrak U^\infty(M),
    (y, V) \in \mathfrak U^\infty(N)}.
\]
Then $M \times N$ becomes an $(m + n)$-manifold, whose underlying topological 
space is the product topological space of $M$ and $N$.

\begin{dfn}
  The manifold $M \times N$ is called the \emph{product of the manifolds $M$
  and $N$}.
\end{dfn}

The two canonical projection maps $\pr_1\colon M \times N \to M, (p, q) \mapsto
p$ and $\pr_2\colon M \times N \to N, (p, q) \mapsto q$ are morphisms.

As simple example of a product manifold is given by $\set R^n$. More precisely,
for any $p$, $q$ with $p + q = n$, the map
\[
  \set R^p \times \set R^q \to \set R^n, ((u_1, \dotsc, u_p), (v_1, \dotsc, v_q))
  \mapsto (u_1, \dotsc, u_p, v_1, \dotsc, v_q)
\]
is a diffeomorphism. (Here, the left hand side is, of course, endowed with the
structure of a product manifold.)

\section{Vectors}
% XXX Put this and the following in a separate chapter.
\label{sec:vectors}

By an \emph{observer}, we mean a curve $\alpha\colon J \to M$ in spacetime $M$,
which might be traced out by a physical body, and relative to which one may
measure events. First and foremost, an observer defines a set of events, namely
the set $\alpha(J)$ of events traced out by the observer. This defines the
relation of being at the same spatial location as the observer: an event $p$ is 
\emph{at the same spatial location as the oberver} if and only if
$p \in \alpha(J)$. The curve $\alpha$, however, is not determined by the set
$\alpha(J)$ of events alone; it does not capture the dynamics of $\alpha$.
For example, if we move infinitesimally from $t$ to $t + \diff t$ on $J$, we
expect that $\alpha(t)$ moves infinitesimally to $\alpha(t) + d \alpha(t)$. In
other words, we expect $\frac{\diff\alpha(t)}{\diff t}$ to be a vector on $M$
tangent to the curve $\alpha(t)$. At the moment, howoever, we have no notion of
such a thing as a vector, which with we could describe dynamics, on a manifold $M$.
The purpose of this section is to remedy this.

The basic idea of defining vectors on a manifold $M$ is the following:
Any point $p$ assigns a value $p^* \phi = \phi(p)$ to each function
$\phi \in \mathcal C^\infty(M)$, and we can reconstruct the point from these
values as the map $p \mapsto p^*$
is injective. Assume that we already have the notion of a vector $v$ at a point $p$.
It would assign a change of values along $v$, a derivative, written $v \cdot \phi$, to each
function $f$. Thus such a vector would give rise to a function
\[
  v\colon \mathcal C^\infty(M) \to \set R, \phi \mapsto v \cdot \phi.
\]
A vector should be reconstructable from this map, so that the functions form
a complete set of observables for vectors. This map should have the
properties of a first-order derivative operator, so we end up with the following
definition:

\begin{dfn}
  A \emph{vector $v$ at a point $p \in M$} is a map
  \[
    v\colon \mathcal C^\infty(M) \to \set R, \phi \mapsto v \cdot \phi,
  \]
  such that the following axioms hold:
  
  \paragraph{Constants}
  For all $c \in \set R$, one has $v \cdot \underline c = 0$.

  \paragraph{Linearity}
  For all $a$, $b \in \set R$ and all $\phi$, $\psi \in \mathcal C^\infty(M)$,
  one has $v \cdot (a \, \phi + b \, \psi) = a \, v \cdot \phi + b \, v \cdot \psi$.
  
  \paragraph{Leibniz rule}
  For all $\phi$, $\psi \in \mathcal C^\infty(M)$, one has
  $v \cdot (\phi \, \psi) = (v \cdot \phi) \, \psi(p) + \phi(p) \, (v \cdot \psi)$.
\end{dfn} 

The real number $v \cdot \phi$ is the \emph{derivative of $\phi$ along $v$}. For
example, the constants axiom says that constants have vanishing derivative along
any vector.

The set of all vectors at a point $p \in M$ is called the \emph{tangent space
of $M$ at $p$} and is denoted by $\Tang_p M$. The tangent space $\Tang_p M$ is
canonically a (real) vector space, where addition and scalar multiplication
are defined ``function-wise'':
\begin{align*}
  0 \cdot \phi & \coloneqq 0,&
  (a \, v) \cdot \phi & \coloneqq a \, (v \cdot \phi), &
  (v + w) \cdot \phi & \coloneqq v \cdot \phi + w \cdot \phi
\end{align*}
for all $v$, $w \in \Tang_p M$, $a \in \set R$ and $\phi \in \mathcal C^\infty(M)$.

The first basic result, which partially shows that the above definition of a 
vector in fact coincides with the intuitive notion, is that the derivative of
a function $\phi$ along a vector at a point $p$ depends only on the values of $\phi$
in a neighborhood of $p$:

\begin{lem}
  Let $G$ be an open neighborhood of a point $p \in M$ such that $\phi|G = \psi|G$
  for two functions $\phi$ and $\psi \in \mathcal C^\infty(M)$. Then
  \[
    \forall v \in \Tang_p M: v \cdot \phi = v \cdot \psi.
  \]
\end{lem}

\begin{proof}
  By \prettyref{lem:paracompact_manifold}, choose a smooth function $\lambda \in
  \mathcal C^\infty$ with $\lambda(p) = 1$ and $\supp \lambda \subseteq G$. Then
  \[
    \begin{split}
      v \cdot \phi - v \cdot \psi
      & = v \cdot (\phi - \psi) = v \cdot \left((1 - \lambda) \, (\phi - \psi)\right) \\
      & = \left(v \cdot (1 - \lambda)\right) \, (\phi(p) - \psi(p))
      + (1 - \lambda(p)) \, \left(v \cdot (\phi - \psi)\right) = 0 + 0 \\
      & = 0.
    \end{split}
  \]
\end{proof}

Given a vector $v$ at $p$ and a function $\phi$ defined in a neighborhood $G$
of $p$, we can therefore speak of the derivative $v \cdot \phi$ of $\phi$ along
$v$: Let $\widehat\phi$ be any extension of $\phi$ by zero away from $p$. As
$\widehat\phi$ and $\phi$ coincide in a neighborhood of $p$, the derivative
\[
  v \cdot \phi \coloneqq v \cdot \widehat\phi
\]
does not depend on the choice of $\widehat\phi$.

With this result we can completely determine the structure of the tangent space
$\Tang_p M$. In fact, we have:

\begin{thm}
  \label{thm:coordinate_basis}
  Let $M$ be an $n$-dimensional manifold. For each point $p \in M$, the tangent
  space $\Tang_p M$ is an $n$-dimensional vector space. More precisely, given
  a chart $(x, U) \in \mathfrak U^\infty(p, M)$, a basis of $\Tang_p M$ is given
  by the vectors
  \begin{equation}
    \label{eq:coordinate_basis}
    \frac{\partial}{\partial x_i}|_p\colon \mathcal C^\infty(M)
    \to \set R, f \mapsto \delta_i ((f|U) \circ x^{-1}) (x(p))
    \eqqcolon \frac{\partial f}{\partial x_i}(p),  
  \end{equation}
  where $i = 1, \dotsc, n$. If $(y, V) \in \mathfrak U^\infty(p, M)$ is another
  chart around $p$, the change of basis is given by
  \begin{equation}
    \label{eq:basis_change}
    \frac{\partial}{\partial x_i}|_p =
    \sum_{j = 1}^n \partial_i ((y_j|U) \circ x^{-1}) (x(p))
    \, \frac{\partial}{\partial y_j}|_p
    =
    \sum_{j = 1}^n \frac{\partial y_j}{\partial x_i}(p) \, 
    \frac{\partial}{\partial y_j}|_p.
  \end{equation}
\end{thm}
The basis $(\frac{\partial}{\partial x_i}|_p)_i$ is called the
\emph{chart's $x$ induced basis of $\Tang_p M$}.

\begin{proof}
  To show that \prettyref{eq:coordinate_basis} defines a vector at $p$ is
  straight-forward and follows from the linearity and product rule of the
  partial differentiation in calculus.
  
  The linear independence of the $n$ vectors defined by
  \prettyref{eq:coordinate_basis} follows directly from the following
  observation: For any $n$-tupel $(a_1, \dotsc, a_n)$ of
  scalars with $\sum_{i = 1}^n a_i \, \frac\partial{\partial x_i}|_p = 0$, one
  calculates
  \[
    \begin{split}
      0 & = \left(\sum_{i = 1}^n a_i \, \frac\partial{\partial x_i}|_p\right) \cdot x_j
      = \sum_{i = 1}^n a_i \, \partial_i (x_j \circ x^{-1}) (x(p)) \\
      & = \sum_{i = 1}^n a_i \, \updelta_{ij}
      = a_j
    \end{split}
  \]
  for all $i = 1$, \dots, $n$.
  
  To show that the $n$ vectors defined by \prettyref{eq:coordinate_basis} span
  the tangent space $\mathrm T_p M$ at $p$, let $\phi$ be any function on $M$. By
  \prettyref{thm:hadamard}, there exist functions $g_1$, \dots, $g_n$ defined
  in a neighborhood of $p$ in $U$ such that
  \[
    \phi = \phi(p) + \sum_{i = 1}^n (x_i - x_i(p)) \, g_i(u)
  \]
  in a neighborhood of $p$ in $\set R^n$. Given any vector $v \in
  \mathrm T_p M$, we thus have
  \[
    \begin{split}
      v \cdot \phi & = v \cdot (\phi(p) + \sum_{i = 1}^n (x_i - x_i(p)) \, g_i) \\
      & = \sum_{i = 1}^n \left((v \cdot (x_i - x_i(p))) \, g_i(p) + (x_i(p) - x_i(p)) \, (v \cdot g_i)\right)
      = \sum_{i = 1}^n (v \cdot x_i) \, g_i(p).
    \end{split}
  \]
  In other words, $v \cdot \phi = \sum_{j = 1}^n (v \cdot x_j) \, \frac\partial{\partial x_j}|_p \cdot \phi$
  for any vector $v \in \mathrm T_p M$. Thus
  \[
    \forall v \in \mathrm T_p M: v = \sum_{i = 1}^n (v \cdot x_j) \, \frac\partial{\partial x_i}|_p,
  \] 
  which is a linear combination of the induced basis.
  
  As this formula also works for the coordinate system $(y, V)$, we get
  \prettyref{eq:basis_change} for $v = \frac\partial{\partial x_i}|_p$.
\end{proof}

The theorem immediately gives the tangent space at a point $p$ of an open
submanifold $U$ of $n$-dimensional space: It is spanned by the $n$ linearly 
independent vectors $\partial_1(p)$, \dots, $\partial_n(p)$ where
\[
  \partial_i(p) \cdot \phi \coloneqq (\partial_i \phi) (p)
\]
for all $i \in \Set{1, \dotsc, n}$. Thus, there is a canonical isomorphism
\[
  \set R^n \to \mathrm T_p M,
  (u_1, \dotsc, u_n) \mapsto \sum_{i = 1}^n u_i \, \partial_i(p).
\]
whose inverse is denoted by
\[
  \diff_p x\colon \mathrm T_p M \to \set R^n.
\]

By the very definition of a vector, we can differentiate functions on a 
manifold, that is morphisms to the manifold $\set R$ along a vector. What is
yet missing is the notion of a derivate of an arbitrary morphism along a vector.
In other words, we are missing a definition of how a morphism maps a vector on
the domain manifold to a vector on the target manifold. This definition is
straight-forward:

Let $f\colon M \to N$ be a morphism of manifolds. For a point $p \in M$, let
$v \in \mathrm T_p M$. The map
\[
  f_* v\colon \mathcal C^\infty(N) \to \mathcal C^\infty(N), \psi \mapsto
  v \cdot (\psi \circ f)
\]
fulfills the linearity and constant axioms and a Leibniz rule as follows
\[
  (f_* v) \cdot (\psi \, \psi') = (v \cdot \phi) \, \psi(f(p)) +
    \phi(f(p)) \, (v \cdot \psi).
\]
In other words, $f_* v$ defines a vector at $f(p)$ defined by the equation
\[
  \forall \psi \in \mathcal C^\infty(N): (f_* v) \cdot \psi = v \cdot (f^{-1} \psi).
\]
The vector $f_* v$ is called the \emph{push-forward of $v$ along $f$}. The
induced linear map on the tangent spaces is denoted by
\[
  \mathrm T_p f\colon \mathrm T_p M \to \mathrm T_p N, v \mapsto f_* v
\]
and called the \emph{tangent map of $f$ at $p$}. That the push-forward
indeed generalizes the derivative of functions follows from the observation that
\[
  \phi_* v = (v \cdot \phi) \, \delta(\phi(p))
\]
for any function $\phi \in \mathcal C^\infty(M)$ viewed as a morphism
$\phi\colon M \to \set R$.

The tangent map of the identity morphism is the identity map, the tangent map
of a composition is the composition of the tangent maps. More precisely, we
have
\[
  \begin{aligned}
    \forall v \in \Tang_p M: &&
    (\id_M)_* v & v, &
    (g \circ f)_* v & = g_* (f_* v),
  \end{aligned}
\]
where $f\colon M \to N$ and $g\colon N \to P$ are two composable morphisms
between manifolds. This observation can be subsumed by saying that the
tangent space is a function from the category of pointed manifolds to the
category of finite-dimensional vector spaces. 

Given a curve $\alpha\colon J \to M$ and a point $t \in J$, which we may view
as a coordinate of time, we can differentiate the curve along $\partial(t)$.
The vector
\[
  \dot\alpha(t) \coloneqq \alpha_* \partial(t)
\]
is called the \emph{derivative} or \emph{velocity of the curve $\alpha$ at $t$}.
The statement about the functoriality of the tangent space can be viewed as the
chain-rule in disguise: in the one-dimensional case we have
\[
  (\alpha \circ c)\spdot(s) = \alpha_* c_* \partial(s)
  = \alpha_* (c'(s) \, \delta(c(s)))
  = c'(s) \, \alpha_* \delta(c(s)) = c'(s) \dot\alpha(c(s)),
\]
where $c\colon I \to J$ is any differential map and $s$ a point in the open
subset $I$ of $\set R$.

Given an observer $\alpha\colon J \to M$ in space-time $M$ and a clock reading
$t \in J$, we get two quantities, namely the event $p \coloneqq \alpha(t)$ and the
derivative $u \coloneqq \dot\alpha(t)$, which is a vector at $p$. In fact, $u$
is all what we need because a vector knows at which point it is attached to.
In the reference frame of the observer $\alpha$, the observer does not move spatially.
In other words, $u$ is a vector pointing in the direction of the observers
future at the event $p$. The length of the vector $u$ can be seen as defining
the observer's unit of time. We thus define an \emph{instantaneous observer}
as just a non-vanishing vector $u$ at any point in space-time.

\section{Tensors}

A vector $v$ at a point $p$ on a manifold $M$ can be seen as a quantity attached to
that point $p$. For example, $v$ could be the velocity of a fluid modelled on
$M$ at the point $p$. Some scalar quantities can also be seen as attached to a point;
for example, the temperature of the fluid at the point $p$. However, not all
quantities attached to points are to be modelled by a scalar or a vector.

As an example, consider a (non-conservative) force field in space, which is
given by a manifold $M$, the set of points in space. The force field is defined
by the work done when a test particle moves through space. If the test particle
is moved infinitesimally from $p$ to $p + \diff p$, an infinitesimal amount
of work $\diff W(p)$ is done, which is a scalar quantity after we have fixed a
unit of work. In other words,
the force field (or, equivalently, the work done by the force field) at a point
$p \in M$ is given by a map
\[
  \diff W(p)\colon \mathrm T_p M \to \set R,
\] 
which associates to each vector $v$ at $p$ the work done when the particle moves
along this vector. By basic principles, this map is linear. In other words,
(after choice of a unit of work) $\diff W(p)$ is an element of the dual space
\[
  \mathrm T^*_p M \coloneqq (\mathrm T_p M)^*
\]
of the vector space $\mathrm T_p M$. For any manifold $M$ and a point $p \in M$,
the vector space $\mathrm T^*_p M$ is called the \emph{cotangent space of $M$
at $p$}. An element of $\mathrm T^*_p M$ is called a \emph{covector} or \emph{linear
form at $p$}. At a given point $p$, a force field is thus modelled by a covector.

An important source of covectors is given by the derivation: Let $\phi \in \mathcal
C^\infty(M)$ be a function on an $n$-dimensional manifold $M$. By definition 
of the vector space structure of a tangent space $\mathrm T_p M$ at a point $p \in M$,
the map
\[
  \diff_p \phi\colon \mathrm T_p M \to \set R,
  v \mapsto v \cdot \phi
\]
is a linear one and thus an element in the cotangent space. The linear form
$\diff_p \phi \in \mathrm T^*_p M$ is the \emph{(total) differential of $\phi$ at
$p$}. Given a function $\phi$ that is only defined in a neighborhood of $p$, the
differential $\diff_p \widehat\phi$ does not depend on the extension $\widehat \phi$
of $\phi$ by zero away from $p$. Thus we can define the \emph{differential
$\diff_p \phi \coloneqq \diff_p \widehat \phi$ of $\phi$ at $p$}. In particular,
every chart $(x, U) \in \mathfrak U^\infty(p, M)$ defines $n$ differentials
$\diff_p x_1$, \dots, $\diff_p x_n$. These differentials form a basis of
$\mathrm T^*_p M$ dual to the basis $\frac{\partial}{\partial x_1}|_p$, \dots,
$\frac{\partial}{\partial x_n}$ of $\mathrm T_p M$, that is
\[
  \forall i, j \in \Set{1, \dotsc, n}:
  \diff_p x_i(\frac{\partial}{\partial x_i}|_p) = \updelta_{ij}.
\]

Besides scalars, vectors and covectors (linear forms), there are even more
complicated objects we need to model physical quantities attached to a point
$p \in M$. Assume that we have a way to define lengths of vectors and angles
between vectors on a manifold $M$. (This is, for example, true if $M$ is
Galilean space where classical mechanics, Newtonian gravity or electrostatics
take place.) After choice of a unit of length, the (square of the)
length of a vector $v \in \Tang_p M$ is given by
\[
  {\norm v}^2 \coloneqq \form{v, v}_p,
\]
where $\form{\cdot, \cdot}_p$ is a positive-definite symmetric bilinear form
\[
  \mathrm T_p M \times \mathrm T_p M \to \set R,
\]
in other words an inner product on $\mathrm T_p M$. (Recall that a bilinear map
is a map that is linear in each of its two arguments. In general, a \emph{multilinear
map} is linear in all of its arguments.) Thus also symmetric bilinear
forms can be attached to a point of space.

As a final example we consider on a manifold $M$ a crystalline solid
with possible dislocations, which are one-dimensional lattice defects. An
infinitesimal parallelogram at a point $p \in M$ is given by two vectors
$u$ and $v \in \mathrm T_p M$ that span the two edges of the parallelogram at
$p$. If one goes around the parallelogram-shaped small loop in the crystalline
solid at $p$, one can roll along the way an idealized perfect version of the
crystalline solid. By this we mean that whenever we move from one
constituent of the actual crystalline solid to the next one while going around
the loop, we take the corresponding move in the idealized crystalline solid. At
the end of the closed loop in the actual crystalline solid, we may not end at
the starting point in the idealized crystalline solid. In general the end point
in the idealized crystalline solid differs from the starting point by a vector
$w$, which we can identify via the crystalline structure with a vector $w$
at $p$ in $M$. The non-vanishing of $w$ is a measurement of the dislocations of
the crystalline solid along the surface element spanned by $u$ and $v$. All
dislocations at $p$ are therefore modelled by a map
\[
  \mathrm T_p M \times \mathrm T_p M \to \mathrm T_p M
\]
that maps $(u, v)$ to $w$. Again by basic principles, we can assume that this
map is linear. This map is an even more general quantity than a scalar, vector,
covector or a bilinear form; it is a bilinear map on $\mathrm T_p M$ taking
values in $\mathrm T_p M$.

The general notion that subsumes all these types of quantities is that of a
tensor:

\begin{dfn}
  A \emph{tensor of type $(k, \ell)$ at a point $p$ on a manifold $M$} is a
  multilinear map
  \[
    (\mathrm T^*_p M)^k \times (\mathrm T_p M)^\ell \to \set R.
  \] 
\end{dfn}
This means, given a tensor $A$ of type $(k, \ell)$ and $k$ covectors
$\alpha_1$, \dots, $\alpha_k$ and $\ell$ vectors $v_1$, \dots, $v_\ell$, one
gets a scalar $A(\alpha_1, \dotsc, \alpha_k, v_1, \dotsc, v_\ell)$, and this
scalar depends linearly on each $\alpha_i$ and on each $v_j$. The set of
tensors at a point $p \in M$ is denoted by $\mathrm T^{(k, \ell)}_p M$. As the
sum of two multilinear maps and the multiplication of a multinear map with a
scalar are again multilinear maps, the set $\mathrm T^{(k, \ell)}_p M$ becomes
a vector space by point-wise addition and scalar multiplication, the \emph{space
of tensors of type $(k, \ell)$ at $p$}.

For example, the work done by a force field at a point $p$ is a tensor of type
$(0, 1)$ (which is nothing but a linear form or covector) and
the inner product $\form{\cdot, \cdot}_p$ from above is a special
tensor (namely a symmetric one) of type $(0, 2)$ at $p$
 A scalar, like the temperature of a fluid at
a point $p$, is also a tensor, namely a tensor of type $(0, 0)$. (By convention,
a multilinear map of no arguments is just an element of the target space, which
is given by the reals in this case.)
The velocity $v$ of a fluid at a point $p$, however, is not immediately seen to be
a tensor of any type as defined above as $v$ is firstly a vector and not a
(multi-)linear map on any space. Nevertheless, this vector induces a linear
map, namely
\[
  v^*\colon \mathrm T^*_p M \to \set R, \alpha \mapsto \alpha(v),
\]
which is a tensor of type $(1, 0)$. From $v^*$ we can reconstruct the vector
$v$ as
\[
  v = \sum_{i = 1}^n v^*(\theta_i) \, u_i
\]
where $u_1$, \dots, $u_n$ is any basis of $\mathrm T_p M$ with dual basis
$\theta_1$, \dots, $\theta_n$. (This observation is
nothing but the well-known fact that canonical linear map
\[
  V \to V^{**}, v \mapsto (v^*\colon \alpha \mapsto \alpha(v))
\]
from a vector space to its double dual is an isomorphism for finite-dimensional
vector spaces.) Therefore, when modelling physical quantities, we won't
distinguish between a vector quantity $v$ and the corresponding tensor quantity
$v^*$ of type $(1, 0)$.

More generally, any $(k + \ell - 1)$-linear map of the form
\[
  L\colon (\mathrm T_p^* M)^{k - 1} \times (\mathrm T_p M)^\ell \to \mathrm T_p M
\]
induces a $(k + \ell)$-linear map
\[
  \begin{aligned}
    L^*\colon & (\mathrm T_p^* M)^k \times (\mathrm T_p M)^\ell \to \mathrm T_p M, \\
    & (\alpha_0, \dotsc, \alpha_{k - 1}, v_1, \dotsc, v_\ell) \mapsto
    \alpha_0(L(\alpha_1, \dotsc, \alpha_{k - 1}, v_1, \dotsc, v_\ell),
  \end{aligned}
\]
that is a tensor if type $(k, \ell)$, and $L$ can be recovered by the formula
\[
  L(\alpha_1, \dotsc, \alpha_{k - 1}, v_1, \dotsc, v_\ell)
  = \sum_{i = 1}^n L^*(\theta_i, \alpha_1, \dotsc, \alpha_{k - 1}, v_1, \dotsc,
  v_\ell) \, u_i.
\]

Thus the dislocation of a crystalline solid at a point $p$, which above
was modelled by a vector-valued bilinear form, is nothing but a tensor of
type $(1, 2)$. 

\section{Vector fields}

As a basic example of a vector on an $n$-dimensional manifold $M$ we gave the velocity vector
of a fluid modelled on $M$ at a point $p$. Often, however, one is not interested
in a single velocity vector but at the totality of all velocity vectors at all
points of $M$. This yields to the following definition:

\begin{dfn}
  A \emph{vector field $X$} on a manifold $M$ is a family $(X(p))_{p \in M}$
  of vectors on $M$ such that $X(p) \in \Tang_p M$ for all $p \in M$ and such
  that
  \[
    X \cdot \phi\colon p \mapsto X(p) \cdot \phi
  \]
  is a (differentiable) function on $M$ for any function $\phi \in \mathcal
  C^\infty(M)$.
\end{dfn}
The latter condition ensures that the vectors $X(p)$ vary smoothly when $p$ is
varied. In general, the velocitites at all points of a fluid are thus modelled
by such a vector field. The set of vector fields is denoted by $\mathfrak X(M)$.
This set becomes an $\mathcal C^\infty(M)$-module by defining addition and
scalar multiplication point-wise.

Every chart $(x, U)$ on $M$ induces $n$ vector fields on the open submanifold
$U$, namely
\[
  \frac{\partial}{\partial x_i}\colon p \mapsto \frac{\partial}{\partial x_i}|_p.
\]
for $i = 1$, \dots, $n$, which form a basis of $\mathrm T_p M$ for all $p \in U$.
Given any system $\phi_1$, \dots, $\phi_n \in \mathcal C^\infty(U)$ of functions
on $U$, the linear combination $\sum_{i = 1}^n \phi_i \, \frac{\partial}{\partial x_i}$
is a vector field on $U$. On the other hand, every vector field on $U$ is of this
form.

If a test particle is put into the fluid on $M$, it will move with the fluid.
The test particle will trace out a curve $\alpha\colon J \to M$ on $M$ such that
the particles velocity will coincide with the fluids velocity at the point of
the particle at all times:
\[
  \forall t \in J: \dot\alpha(t) = X(\alpha(t))
\]
where $X$ is the vector field describing the fluid's velocity with respect to
a chosen unit of time. A curve for which $\dot \alpha = X \circ \alpha$ holds
for a given vector field $X$ on a manifold $M$ is called an \emph{integral
curve of $X$}. It is the curve that is traced out by following the ``flow'' of
the vector field. The question of existence of integral curves is handled by
the following theorem:

\begin{thm}
  Let $M$ be a manifold and $X \in \mathfrak X(M)$.
  \paragraph{Existence of maximal integral curves}
  For each pair $(t_0, p) \in \set R \times M$ there exists an integral curve
  $\alpha\colon J \to M$ with $\alpha(t_0) = p$ that is maximal and unique in the
  following sense: Is $\widetilde\alpha\colon \widetilde J \to M$ any integral
  curve on $X$ with $\widetilde \alpha(t_0) = p$, then $\widetilde J \subseteq
  J$ and $\alpha|\widetilde J = \widetilde \alpha$. The integral curve $\alpha$
  is the \emph{maximal integral curve of $X$ to the initial condition $(t_0, p)$}.
  The maximal integral curve of $X$ to the initial condition $(0, p)$ is denoted
  by $\alpha_p \coloneqq \alpha_p^X\colon J_p \to M$.
  
  \paragraph{Translation invariance of integral curves}
  If $\alpha\colon J \to M$ is any integral curve on $M$ and $s \in \set R$, then
  $\beta\colon -s + J \to M, t \mapsto \alpha(t + s)$ is an integral curve.
  
  \paragraph{Maximal flow}
  The set
  \[
    D \coloneqq D^X \coloneqq \bigcup_{p \in M} (J_p \times \Set p)
  \]
  is an open neighborhood of $\Set 0 \times M$ in $\set R \times M$ and
  \[
    \Phi \coloneqq \Phi^X\colon D \to M, (t, p) \mapsto \alpha_p(t)
  \]
  is a morphism, which is called the \emph{maximal flow of $X$}.

  \paragraph{Local $1$-parameter group}
  For $t \in \set R$, let $D_t$ be the open subset
  \[ 
    D_t \coloneqq \Set{p \in M : (t, p) \in D} \subseteq M.
  \]
  The morphism
  \[
    \Phi_t \coloneqq \Phi_t^X\colon D_t \to M, p \mapsto \Phi(t, p)
  \]
  is a diffeomorphism onto $D_{-t}$. (In particular $D_{-t} \neq \emptyset$
  whenever $D_t \neq \emptyset$.)
  One has $D_0 = M$ and $\Phi_0 = \id_M$. Moreover one has
  \begin{gather*}
    \forall t, s \in \set R: 0 \leq t < s \implies D_s \subseteq D_t,\\
    M = \bigcup_{t \in \set R_+} D_t, \\
    \intertext{and}
    \forall t, s \in \set R, p \in D_s:
      \left\{\begin{gathered}
        \Phi_s(p) \in D_t \iff p \in D_{t + s},\\
        p \in D_{t + s} \implies \Phi_t \circ \Phi_s(p) = \Phi_{t + s}(p).
      \end{gathered}\right.
  \end{gather*}
  The family $(\Phi^X_t)_{t \in \set R}$ is the \emph{dynamical system generated
  by $X$}.
\end{thm}

% \begin{proof} \end{proof}
% lie bracket

\section{Tensor fields}

\section{Problems}

\begin{xca}
  Prove that every $n$-dimensional atlas of a set $M$ is contained in a unique
  maximal $n$-dimensional atlas of $M$.
\end{xca}

\begin{xca}
  Prove that the system of open subsets of a premanifold $M$ is a topology on
  the underlying set of $M$.
\end{xca}

\begin{xca}
  Let $(x, U)$ be a chart of an $n$-dimensional premanifold $M$. Show that a 
  subset $V$ of $U$ is open in $M$ if and only if $x(V)$ is open in $\set R^n$.
  Conclude that $x\colon U \to \set R^n$ is a continuous map.
\end{xca}

\begin{xca}
  Prove that an open submanifold of a manifold is in fact a manifold.
\end{xca}

\begin{xca}
  Prove the following:
  Let $M$ be a manifold and $\phi\colon M \to \set R$ a map. Assume that for each
  point $p \in M$ there exists a chart $(x, U) \in \mathfrak U^\infty(p, M)$
  such that $\phi \circ x^{-1}\colon x(U) \to \set R$ is smooth. Then $\phi$
  is a (smooth) function.
\end{xca}

\begin{xca}
  Prove that every function $f$ on an $n$-dimensional manifold $M$ is a
  continuous map $f\colon M \to \set R$ for the underlying topologies of $M$ and
  $\set R$.
\end{xca}

\begin{xca}
  Prove that the functions on a manifold $M$ form a \emph{sheaf of algebras},
  that is prove that $\mathcal C^\infty(U)$ is closed under constants,
  addition and multiplication for every open subset $U$ of $M$, that the restrictions
  $\mathcal C^\infty(U) \to \mathcal C^\infty(V)$ are homomorphisms of algebras
  for every inclusion $V \subseteq U$ of open subsets of $M$, and that 
  for every open cover $(U_i)_{i \in I}$ of $M$ and functions
  $\phi_i \in \mathcal C^\infty(M)$ one has
  \[
    \left(\forall i, j \in I : \phi_i|U_i \cap U_j = \phi_j|U_i \cap U_j\right)
    \implies \exists! \phi \in \mathcal C^\infty(M) \, \forall i \in I :
    \phi|U_i = \phi_i.
  \]
\end{xca}

\begin{xca}
  Prove that a map $f\colon M \to N$ between manifolds is a morphism if and only if
  \[
    \forall \psi \in \mathcal C^\infty(M):
    f^{-1} \psi \in \mathcal C^\infty(N).
  \]
\end{xca}

\begin{xca}
  Prove that the composition $g \circ f$ or two morphisms $f\colon M \to N$
  and $g\colon N \to P$ between manifolds is again a morphism.
\end{xca}

\begin{xca}
  Prove that the inclusion morphism $i\colon U \to M$ for any open submanifold
  $U$ of a manifold $M$ is in fact a morphism.
\end{xca}

\begin{xca}
  Let $M$ and $N$ be two manifolds. Show that 
  \[
    \Set{(x \times y, U \times V): (x, U) \in \mathfrak U^\infty(M),
      (y, V) \in \mathfrak U^\infty(N)}.
  \]
  is an atlas of the cartesian product $M \times N$.
\end{xca}

\begin{xca}
  Show that the underlying topological space of the product $M \times N$ of
  two manifolds $M$ and $N$ coincides of the topological product of the topological 
  spaces underlying $M$ and $N$.
\end{xca}

\begin{xca}
  Prove that the projections $\pr_1\colon M \times N \to M$ and $\pr_2\colon
  M \times N \to N$ defined on a product manifold $M \times N$ are in fact
  morphisms.
\end{xca}

\begin{xca}
  Let $f\colon M \to N$ be a morphism between manifolds and $p \in M$ a point.
  Prove that
  \[
    f_* v\colon \mathcal C^\infty(N) \to \mathcal C^\infty(N),
    \psi \mapsto v \cdot (\psi \circ f)
  \]
  in fact defines a vector at $f(p)$.
\end{xca}

\begin{xca}
  Let $f\colon M \to N$ be a morphism between manifolds and $p \in M$ a point.
  Proof that the tangent map at $p$ is a linear map
  \[
    \mathrm T_f M\colon \mathrm T_p M \to \mathrm T_{f(p)} N, v \mapsto f_* v.
  \]  
\end{xca}

\begin{xca}
  Let $M$ be a manifold and $p \in M$ be a point. Show that
  \[
    \forall \phi \in \mathcal C^\infty(M):
    \phi_* v = (v \cdot \phi) \, \delta(\phi(p)).
  \]
  for all $v \in \mathrm T_p M$.
\end{xca}

\begin{xca}
  Let $M$ be a manifold and $p \in M$ be a point. Show that
  \[
    \begin{aligned}
      \forall v \in \Tang_p M: &&
      (\id_M)_* v & v, &
      (g \circ f)_* v & = g_* (f_* v),
    \end{aligned}
  \]
  where $f\colon M \to N$ and $g\colon N \to P$ are two (composable) morphisms
  between manifolds.
\end{xca}

\begin{xca}
  Let $\phi$ and $\psi$ be two functions on a manifold $M$
  that coincides in a neighborhood of a point $p \in M$. Why is
  $\diff_p \phi = \diff_p \psi$?  
\end{xca}

\begin{xca}
  Let $M$ be a manifold and $p \in M$ be a point. Show that for any chart
  $(x, U) \in \mathfrak U^\infty(p, M)$, the system
  \[
    \diff_p x_1, \dotsc, \diff_p x_n
  \]
  is a dual basis to the basis
  \[
    \frac{\partial}{\partial x_1}|_p,
    \dotsc, \frac{\partial}{\partial x_n}|_p.
  \]
\end{xca}

\begin{xca}
  Let $v \in \mathrm T_p M$ be a vector at a point $p$ on a manifold $M$. Let
  $(x, U) \in \mathfrak U^\infty(p, M)$ be any chart near $p$. Let
  $v^*\colon \alpha \mapsto \alpha(v)$ be the tensor of type $(1, 0)$ at $p$
  induced by $p$.
  Show that
  \[
    v = \sum_{i = 1}^n v_i \, \frac{\partial}{\partial x_i}|_p,
  \]
  where $v_i \coloneqq v^*(\diff_p x_i)$ for $i \in \Set{1, \dotsc, n}$.
\end{xca}

\begin{xca}
  Let $(x, U)$ be a chart of an $n$-dimensional manifold $M$. Show that every
  vector field $X \in \mathfrak X(U)$ uniquely defines a system $\phi_1$, \dots,
  $\phi_n \in \mathcal C^\infty(U)$ such that
  \[
    X = \sum_{i = 1}^n \phi_i \, \frac{\partial}{\partial x_i}.
  \]
\end{xca}
