\chapter{Analysis}

\section{Bump functions}
\label{sec:bump_functions}

\begin{thm}
  \label{thm:bump_function}
  For any two real numbers $0 < r < R$ there exists a smooth
  function $\phi\colon \set R^n \to \set R$ with
  \begin{align*}
    \forall p \in \set R^n: \phi(p) & \ge 0, &
    \forall p \in \set R^n: \norm p \leq r \implies \phi(p) & = 1, &
    \forall p \in \set R^n: \norm p \geq R \implies \phi(p) & = 0.
  \end{align*}
\end{thm}

A function as in the theorem is called a \emph{bump function}.

\section{Hadamard's lemma}

A subset $U$ of $n$-dimensional Euclidean space $\set R^n$ is \emph{star-shaped with
respect to a point $p \in U$} if
\[
  \forall x \in U \forall 0 \leq t \leq 1 :
  (1 - t) \, a + t \, x \in U.
\]
Any neighborhood of a point $p \in \set R^n$ contains an open neighborhood
star-shaped with respect to the point $p$ as the standard $\epsilon$-neighborhoods
are star-shaped with respect to their center (in fact, with respect to any point
of their interior).

\begin{thm}
  \label{thm:hadamard}
  Let $\phi$ be a smooth function defined on an open subset $U$ of $n$-dimensional
  Euclidean space $\set R^n$ that is star-shaped with respect to a point $a \in U$.
  Then there exist smooth functions $g_1$, \dots, $g_n$ on $U$ such that
  \[
    \forall x \in U: \phi(x) = \phi(a) + \sum_{i = 1}^n (x_i - a_i) \, g_i(x).
  \]
\end{thm}
