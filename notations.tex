\chapter*{Notations}

\paragraph{Standard sets}
We use the following notations for the standard sets: The set of natural
numbers (which includes $0$, by definition) is denoted by $\set N_0$, the set
of integers by $\set Z$, the set of rational numbers by $\set Q$, the set of
real numbers by $\set R$ and the set of complex numbers by $\set C$.

The set of the positive real numbers is denoted by a subscript: $\set R_+$. 

\paragraph{Real numbers}

For any interval $J \subseteq \set R$ and any $s \in \set R$, one defines the interval
\[
  - s + J \coloneqq \Set{t \in \set R : s + t \in J}.
\]

\paragraph{Linear algebra}

In an $n$-dimensional vector space, the Kronecker symbol is a scalar defined by 
\[
  \updelta_{ij} \coloneqq \begin{cases}
    1 & \text{if $i = j$} \\
    0 & \text{otherwise}
  \end{cases}
\]
for all $i$, $j \in \Set{1, \dotsc, n}$.

Given a vector space $V$, its \emph{dual space $V^*$} is the set $\Hom(V, \set R)$
of linear maps from $V$ to the scalars $\set R$, endowed with a vector space
structure by defining addition and scalar multiplication point-wise.

\paragraph{Maps}

By a \emph{differential map} we will always mean of map of class $\mathcal C^\infty$
that is a smooth map. \emph{Differentiability} thus means the existence of
continuous derivates to all orders.

The term \emph{function} will be reserved for smooth maps with values in $\set R$.
Thus a function is always a smooth function.

\paragraph{Cartesian space}
The standard basis formed by $e_1$, \dots, $e_n$ of $n$-dimensional space $\set R^n$ is
a basis of the underlying vector space such that $v = \sum_{i = 1}^n v_i \, e_i
$ for each $v = (v_1, \dotsc, v_n) \in \set R^n$. 

The standard Euclidean norm of $n$-dimensional space $\set R^n$ is denoted
by
\[
    \norm v \coloneqq \sqrt{\sum_{i = 1}^n v_i^2}
\]
for $v = (v_1, \dotsc, v_n) \in \set R^n$.

The partial derivative of a function $\phi$ defined on an open subset of $\set R^n$
in direction $i$ is denoted by $\partial_i$, that is
\[
  \partial_i \phi\colon v \mapsto \frac{\partial \phi(v + t \, e_i)}{\partial t}|_{t = 0}
\]
for $i = 1$, \dots, $n$. In case of $n = 1$, we set $\partial \coloneqq \partial_1$.
